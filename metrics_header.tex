

\usepackage{libertine}

% If you can't see cyrillic letters in R-studio choose
% File-Reopen with encoding
% utf8 is the preferred encoding

\usepackage{setspace}
\setstretch{1}                          % Межстрочный интервал

\flushbottom                            % Эта команда заставляет LaTeX чуть растягивать строки, чтобы получить идеально прямоугольную страницу
\righthyphenmin=2                       % Разрешение переноса двух и более символов
\widowpenalty=300                     % Небольшое наказание за вдовствующую строку (одна строка абзаца на этой странице, остальное --- на следующей)
\clubpenalty=3000                     % Приличное наказание за сиротствующую строку (омерзительно висящая одинокая строка в начале страницы)
\setlength{\parindent}{1.5em}           % Красная строка.
%\captiondelim{. }
\setlength{\topsep}{0pt}
\emergencystretch=2em

% делаем короче интервал в списках
\setlength{\itemsep}{0pt}
\setlength{\parskip}{0pt}
\setlength{\parsep}{0pt}


\usepackage{framed} % для рамок и черты слева от минитеории, \leftbar


%\usepackage{showkeys} % показывать метки в готовом pdf

% \usepackage{etex} % расширение классического tex
% в частности позволяет подгружать гораздо больше пакетов, чем мы и займёмся далее

\usepackage{verbatim} % для многострочных комментариев
\usepackage{makeidx} % для создания предметных указателей

\usepackage{setspace}
\usepackage{amsmath, amsfonts, amssymb, amsthm}
\usepackage{mathrsfs} % sudo yum install texlive-rsfs
\usepackage{dsfont} % sudo yum install texlive-doublestroke
\usepackage{array, multicol, multirow, bigstrut} % sudo yum install texlive-multirow
\usepackage{indentfirst} % установка отступа в первом абзаце главы
\usepackage{bm}
\usepackage{bbm} % шрифт с двойными буквами

\usepackage{dcolumn} % центрирование по разделителю для apsrtable

% создание гиперссылок в pdf
\usepackage[unicode, colorlinks=true, urlcolor=blue, hyperindex, breaklinks]{hyperref}

% свешиваем пунктуацию
% теперь знаки пунктуации могут вылезать за правую границу текста, при этом текст выглядит ровнее
\usepackage{microtype}

\usepackage{textcomp}  % Чтобы в формулах можно было русские буквы писать через \text{}

% размер листа бумаги
%\usepackage[paperwidth=145mm,paperheight=215mm,
%height=182mm,width=113mm,top=20mm,includefoot]%{geometry}
\usepackage[paper=a4paper, top=15mm, bottom=13.5mm, left=16.5mm, right=13.5mm, includefoot]{geometry}

\usepackage{xcolor}


\usepackage{float, longtable}
% \usepackage{soulutf8} % obsolete
\usepackage{soul}

\usepackage{enumitem} % дополнительные плюшки для списков
%  например \begin{enumerate}[resume] позволяет продолжить нумерацию в новом списке

\usepackage{mathtools} % для \DeclarePairedDelimiter
\usepackage{cancel,xspace} % sudo yum install texlive-cancel


\usepackage{numprint} % sudo yum install texlive-numprint
\npthousandsep{,}\npthousandthpartsep{}\npdecimalsign{.}


\usepackage{subfigure} % для создания нескольких рисунков внутри одного

\usepackage{tikz, pgfplots} % язык для рисования графики из latex'a
\usepackage{tikz-3dplot} % для 3d графиков
\usetikzlibrary{trees} % tikz-прибамбас для рисовки деревьев
\usepackage{tikz-qtree} % альтернативный tikz-прибамбас для рисовки деревьев
\usetikzlibrary{arrows} % tikz-прибамбас для рисовки стрелочек подлиннее

\usepackage{todonotes} % для вставки в документ заметок о том, что осталось сделать
% \todo{Здесь надо коэффициенты исправить}
% \missingfigure{Здесь будет Последний день Помпеи}
% \listoftodos --- печатает все поставленные \todo'шки


% более красивые таблицы
\usepackage{booktabs}
% заповеди из докупентации:
% 1. Не используйте вертикальные линни
% 2. Не используйте двойные линии
% 3. Единицы измерения - в шапку таблицы
% 4. Не сокращайте .1 вместо 0.1
% 5. Повторяющееся значение повторяйте, а не говорите "то же"





\usepackage{fontspec} % что-то про шрифты?
\usepackage{polyglossia} % русификация xelatex
% \usepackage{physics} % куча приятных сокращений
% не рекомендуют его использовать
% https://tex.stackexchange.com/questions/471532/alternatives-to-the-physics-package
\DeclarePairedDelimiter{\norm}\lVert\rVert
\DeclareMathOperator{\tr}{trace}


% \usepackage{minted}
\usepackage{listings}
 \lstset{
  basicstyle=\ttfamily,
  columns=fullflexible,
  keepspaces=true,
}





% download "Linux Libertine" fonts:
% http://www.linuxlibertine.org/index.php?id=91&L=1
% \setmainfont{Linux Libertine O} % or Helvetica, Arial, Cambria
% why do we need \newfontfamily:
% http://tex.stackexchange.com/questions/91507/
% \newfontfamily{\cyrillicfonttt}{Linux Libertine O}


\DeclareMathOperator{\rk}{rank}


\DeclareMathOperator{\Corr}{Corr}
\DeclareMathOperator{\sCorr}{sCorr}
\DeclareMathOperator{\sCov}{sCov}
\DeclareMathOperator{\sVar}{sVar}
\DeclareMathOperator{\Cov}{Cov}
\DeclareMathOperator{\Var}{Var}
\DeclareMathOperator{\E}{\mathbb E}
\DeclareMathOperator{\hVar}{\widehat{\Var}}
\DeclareMathOperator{\hCorr}{\widehat{\Corr}}
\DeclareMathOperator{\hCov}{\widehat{\Cov}}
\DeclareMathOperator{\Lin}{Lin}
\DeclareMathOperator{\Linp}{Lin^{\perp}}
\DeclareMathOperator{\Med}{Med}

\let\L\relax
\DeclareMathOperator{\L}{L} %% лаг

\let\P\relax
\DeclareMathOperator{\P}{\mathbb P}

\DeclareMathOperator*{\plim}{plim}
\DeclareMathOperator{\MSE}{MSE}

\newcommand \R{\mathbb R}
\newcommand \N{\mathbb N}
\newcommand \Q{\mathbb Q}
\newcommand \Z{\mathbb Z}

\newcommand \hb{\hat{\beta}}
\newcommand \hs{\hat{\sigma}}
\newcommand \hy{\hat{y}}
\newcommand \hY{\hat{Y}}
\newcommand \htheta{\hat{\theta}}

\newcommand \e{\varepsilon}
\newcommand \he{\hat\e}
\newcommand \vone{\vec{1}}

\newcommand \cN{\mathcal{N}}


% вместо горизонтальной делаем косую черточку в нестрогих неравенствах
\renewcommand{\le}{\leqslant}
\renewcommand{\ge}{\geqslant}
\renewcommand{\leq}{\leqslant}
\renewcommand{\geq}{\geqslant}


\AddEnumerateCounter{\asbuk}{\russian@alph}{щ} % для списков с русскими буквами

\begin{translation-ru}
\setlist[enumerate, 1]{label=\asbuk*),ref=\asbuk*} % списки уровня 1 будут буквами а) б) ...
\end{translation-ru}

\begin{translation-en}
\setlist[enumerate, 1]{label=\alph*),ref=\alph*} % списки уровня 1 будут буквами а) б) ...
\end{translation-en}
    


% this magick is to solve problem that appeared after update of texlive 2018 to texlive 2020
% https://tex.stackexchange.com/questions/511341/the-error-occurred-after-the-last-update
\makeatletter
\def\nobreak{\penalty\@M}
\makeatother


\usepackage[bibencoding = auto,
style = alphabetic,
backend = biber,
citestyle = alphabetic,
sorting = none]{biblatex}

\addbibresource{metrics_pro.bib}

\begin{translation-ru}
\title{Заметки к семинарам по метрике}    
\end{translation-ru}

\begin{translation-en}
\title{Econometrics: class notes}    
\end{translation-en}
    

\author{\url{https://github.com/bdemeshev/metrics_pro}}
\date{\today}



\usepackage{answers} % дележка условий и ответов

%\newtheorem{problem}{Задача}
%\numberwithin{problem}{section}

\Newassociation{sol}{solution}{solution_file}
% sol — имя окружения внутри задач
% solution — имя окружения внутри solution_file
% solution_file — имя файла в который будет идти запись решений
% можно изменить далее по ходу
\Opensolutionfile{solution_file}[all_solutions]
% в квадратных скобках фактическое имя файла


% магия для автоматических гиперссылок задача-решение
\newlist{myenum}{enumerate}{3}
% \newcounter{problem}[chapter] % нумерация задач внутри глав
\newcounter{problem}[section]

\newenvironment{problem}%
{%
\refstepcounter{problem}%
%  hyperlink to solution
     \hypertarget{problem:{\thesection.\theproblem}}{} % нумерация внутри глав
     % \hypertarget{problem:{\theproblem}}{}
     \Writetofile{solution_file}{\protect\hypertarget{soln:\thesection.\theproblem}{}}
     %\Writetofile{solution_file}{\protect\hypertarget{soln:\theproblem}{}}
     \begin{myenum}[label=\bfseries\protect\hyperlink{soln:\thesection.\theproblem}{\thesection.\theproblem},ref=\thesection.\theproblem]
     % \begin{myenum}[label=\bfseries\protect\hyperlink{soln:\theproblem}{\theproblem},ref=\theproblem]
     \item%
    }%
    {%
    \end{myenum}}
% для гиперссылок обратно надо переопределять окружение
% это происходит непосредственно перед подключением файла с решениями





\setcounter{tocdepth}{1} % в оглавление оставляем уровень 1

%\usepackage[titles]{tocloft}
\usepackage{tocbasic}
\DeclareTOCStyleEntry[
  beforeskip=.1em plus 1pt,% default is 1em plus 1pt
  linefill=\bfseries\TOCLineLeaderFill,% точечки
  pagenumberformat=\textbf
]{tocline}{section}


\pgfplotsset{compat=1.18}
