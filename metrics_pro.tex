\documentclass[11pt, a4paper]{article}

% If you can't see cyrillic letters in R-studio choose
% File-Reopen with encoding
% utf8 is the preferred encoding


%%%%%%%%%%%%%%%%%%%%%%%  Загрузка пакетов  %%%%%%%%%%%%%%%%%%%%%%%%%%%%%%%%%%
% кусок от урсса
%\usepackage[60x90,headers,11pt]{format}

%\textheight=494pt%
%\textwidth=322pt%
%
%\oddsidemargin=0pt%
%\evensidemargin=0pt
%\topmargin=-1pt \headsep=14pt \headheight=22pt \voffset=-28pt
%\hoffset=-50pt


\clubpenalty=10000
\widowpenalty=10000

%\overfullrule=5pt
%\hfuzz=1.5mm
%\baselineskip=12pt plus 0.18pt minus 0.1pt


%\pagestyle{headings}
% конец куска от урсса




% специальная версия для knitr'а. Исключает graphicx

%\usepackage{showkeys} % показывать метки в готовом pdf

\usepackage{etex} % расширение классического tex
% в частности позволяет подгружать гораздо больше пакетов, чем мы и займёмся далее

%\usepackage{mathtext} % русские буквы в формулах? (и без неё работает)
% Например, $x_{\text{один}}$

%\usepackage{cmap} % для поиска русских слов в pdf --- теперь работает без этого
% а с cmap не работает печать на принтер ;)
\usepackage{verbatim} % для многострочных комментариев
\usepackage{makeidx} % для создания предметных указателей

\usepackage{setspace}
\usepackage{amsmath, amsfonts, amssymb, amsthm}
\usepackage{mathrsfs} % sudo yum install texlive-rsfs
\usepackage{dsfont} % sudo yum install texlive-doublestroke
\usepackage{array, multicol, multirow, bigstrut} % sudo yum install texlive-multirow
\usepackage{indentfirst} % установка отступа в первом абзаце главы
\usepackage{bm}
\usepackage{bbm} % шрифт с двойными буквами
%\usepackage[perpage]{footmisc}

\usepackage{dcolumn} % центрирование по разделителю для apsrtable

% создание гиперссылок в pdf
\usepackage[unicode, colorlinks=true, urlcolor=blue, hyperindex, breaklinks]{hyperref}

% свешиваем пунктуацию
% теперь знаки пунктуации могут вылезать за правую границу текста, при этом текст выглядит ровнее
\usepackage{microtype}

\usepackage{textcomp}  % Чтобы в формулах можно было русские буквы писать через \text{}

% размер листа бумаги
%\usepackage[paperwidth=145mm,paperheight=215mm,
%height=182mm,width=113mm,top=20mm,includefoot]%{geometry}
\usepackage[paper=a4paper, top=15mm, bottom=13.5mm, left=16.5mm, right=13.5mm, includefoot]{geometry}

\usepackage{xcolor}


\usepackage{float, longtable}
\usepackage{soulutf8}

\usepackage{enumitem} % дополнительные плюшки для списков
%  например \begin{enumerate}[resume] позволяет продолжить нумерацию в новом списке

\usepackage{mathtools}
\usepackage{cancel,xspace} % sudo yum install texlive-cancel

% \usepackage{minted} % display program code with syntax highlighting
% требует установки pygments и python

\usepackage{numprint} % sudo yum install texlive-numprint
\npthousandsep{,}\npthousandthpartsep{}\npdecimalsign{.}


\usepackage{subfigure} % для создания нескольких рисунков внутри одного

\usepackage{tikz, pgfplots} % язык для рисования графики из latex'a
\usepackage{tikz-3dplot} % для 3d графиков
\usetikzlibrary{trees} % tikz-прибамбас для рисовки деревьев
\usepackage{tikz-qtree} % альтернативный tikz-прибамбас для рисовки деревьев
\usetikzlibrary{arrows} % tikz-прибамбас для рисовки стрелочек подлиннее

\usepackage{todonotes} % для вставки в документ заметок о том, что осталось сделать
% \todo{Здесь надо коэффициенты исправить}
% \missingfigure{Здесь будет Последний день Помпеи}
% \listoftodos --- печатает все поставленные \todo'шки


% более красивые таблицы
\usepackage{booktabs}
% заповеди из докупентации:
% 1. Не используйте вертикальные линни
% 2. Не используйте двойные линии
% 3. Единицы измерения - в шапку таблицы
% 4. Не сокращайте .1 вместо 0.1
% 5. Повторяющееся значение повторяйте, а не говорите "то же"




\usepackage{fontspec} % что-то про шрифты?
\usepackage{polyglossia} % русификация xelatex

\setmainlanguage{russian}
\setotherlanguages{english}

% download "Linux Libertine" fonts:
% http://www.linuxlibertine.org/index.php?id=91&L=1
\setmainfont{Linux Libertine O} % or Helvetica, Arial, Cambria
% why do we need \newfontfamily:
% http://tex.stackexchange.com/questions/91507/
\newfontfamily{\cyrillicfonttt}{Linux Libertine O}



%\usepackage{asymptote} % пакет для рисовки графики, должен идти после graphics
% но мы переходим на tikz :)

%\usepackage{sagetex} % для интеграции с Sage (вероятно тоже должен идти после graphics)





%%%%%%%%%%%%%%%%%%%%%%%  ПАРАМЕТРЫ  %%%%%%%%%%%%%%%%%%%%%%%%%%%%%%%%%%
\setstretch{1}                          % Межстрочный интервал
\flushbottom                            % Эта команда заставляет LaTeX чуть растягивать строки, чтобы получить идеально прямоугольную страницу
\righthyphenmin=2                       % Разрешение переноса двух и более символов
%\pagestyle{plain}                       % Нумерация страниц снизу по центру.
%\widowpenalty=300                     % Небольшое наказание за вдовствующую строку (одна строка абзаца на этой странице, остальное --- на следующей)
%\clubpenalty=3000                     % Приличное наказание за сиротствующую строку (омерзительно висящая одинокая строка в начале страницы)
\setlength{\parindent}{1.5em}           % Красная строка.
%\captiondelim{. }
\setlength{\topsep}{0pt}
%%%%%%%%%%%%%%%%%%%%%%%%%%%%%%%%%%%%%%%%%%%%%%%%%%%%%%%%%%%%%%%%%%%%%%



%%%%%%%% Это окружение, которое выравнивает по центру без отступа, как у простого center
\newenvironment{center*}{%
  \setlength\topsep{0pt}
  \setlength\parskip{0pt}
  \begin{center}
}{%
  \end{center}
}
%%%%%%%%%%%%%%%%%%%%%%%%%%%%%%%%%%%%%%%%%%%%%%%%%%%%%%%%%%%%%%%%%%%%%%


%%%%%%%%%%%%%%%%%%%%%%%%%%% Правила переноса  слов
\hyphenation{ }
%%%%%%%%%%%%%%%%%%%%%%%%%%%%%%%%%%%%%%%%%%%%%%%%%%%%%%%%%%%%%%%%%%%%%%

\emergencystretch=2em


% DEFS
\newcommand \mbf{\mathbf}
\newcommand \msf{\mathsf}
\newcommand \mbb{\mathbb}
\newcommand \tbf{\textbf}
\newcommand \tsf{\textsf}
\newcommand \ttt{\texttt}
\newcommand \tbb{\textbb}

\newcommand \wh{\widehat}
\newcommand \wt{\widetilde}
\newcommand \ol{\overline}
\newcommand \cd{\cdot}
\newcommand \fr{\frac}
\newcommand \bs{\backslash}
\newcommand \lims{\limits}
\DeclareMathOperator{\dist}{dist}
\DeclareMathOperator{\VC}{VCdim}
\DeclareMathOperator{\card}{card}
\DeclareMathOperator{\sign}{sign}
\DeclareMathOperator{\sgn}{sign}
\DeclareMathOperator{\Tr}{\mbf{Tr}}
\DeclareMathOperator{\tr}{tr}


\newcommand \xfs{(x_1,\ldots,x_{n-1})}
\DeclareMathOperator*{\argmin}{arg\,min}
\DeclareMathOperator*{\amn}{arg\,min}
\DeclareMathOperator*{\amx}{arg\,max}
\DeclareMathOperator{\trace}{tr}
\DeclareMathOperator{\rk}{rk}


\DeclareMathOperator{\Corr}{Corr}
\DeclareMathOperator{\sCorr}{sCorr}
\DeclareMathOperator{\sCov}{sCov}
\DeclareMathOperator{\sVar}{sVar}

\DeclareMathOperator{\Cov}{Cov}
\DeclareMathOperator{\Var}{Var}
\DeclareMathOperator{\corr}{Corr}
\DeclareMathOperator{\cov}{Cov}
\DeclareMathOperator{\var}{Var}
\DeclareMathOperator{\bin}{Bin}
\DeclareMathOperator{\Bin}{Bin}
\DeclareMathOperator{\rang}{rang}
\DeclareMathOperator*{\plim}{plim}
\DeclareMathOperator{\MSE}{MSE}

\providecommand{\iff}{\Leftrightarrow}
\providecommand{\hence}{\Rightarrow}

\newcommand \ti{\tilde}
\newcommand \wti{\widetilde}

\newcommand \mA{\mathcal{A}}
\newcommand \mB{\mathcal{B}}
\newcommand \mC{\mathcal{C}}
\newcommand \mE{\mathcal{E}}
\newcommand \mF{\mathcal{F}}
\newcommand \mH{\mathcal{H}}
\newcommand \mL{\mathcal{L}}
\newcommand \mN{\mathcal{N}}
\newcommand \mU{\mathcal{U}}
\newcommand \mV{\mathcal{V}}
\newcommand \mW{\mathcal{W}}


\newcommand \R{\mbb R}
\newcommand \N{\mbb N}
\newcommand \Z{\mbb Z}
\newcommand{\E}{\mathbb{E}}
\newcommand \D{\msf{D}}
\newcommand \I{\mbf{I}}

\newcommand \QQ{\mbb Q}
\newcommand \RR{\mbb R}
\newcommand \NN{\mbb N}
\newcommand \ZZ{\mbb Z}
\newcommand \PP{\mbb P}


\newcommand \dt{\delta}
\newcommand{\e}{\varepsilon}
\newcommand \ga{\gamma}
\newcommand \kp{\varkappa}
\newcommand \la{\lambda}
\newcommand \sg{\sigma}
\newcommand \sgm{\sigma}
\newcommand \ve{\varepsilon}
\newcommand \Dt{\Delta}
\newcommand \La{\Lambda}
\newcommand \Sgm{\Sigma}
\newcommand \Sg{\Sigma}
\newcommand \Tt{\Theta}
\newcommand \Om{\Omega}
\newcommand \om{\omega}

%\newcommand{\p}{\partial}

\newcommand \lbr{\linebreak}
\newcommand \vsi{\vspace{0.1cm}}
\newcommand \vsii{\vspace{0.2cm}}
\newcommand \vsiii{\vspace{0.3cm}}
\newcommand \vsiv{\vspace{0.4cm}}
\newcommand \vsv{\vspace{0.5cm}}
\newcommand \vsvi{\vspace{0.6cm}}
\newcommand \vsvii{\vspace{0.7cm}}
\newcommand \vsviii{\vspace{0.8cm}}
\newcommand \vsix{\vspace{0.9cm}}
\newcommand \VSI{\vspace{1cm}}
\newcommand \VSII{\vspace{2cm}}
\newcommand \VSIII{\vspace{3cm}}

\newcommand{\bls}[1]{\boldsymbol{#1}}
\newcommand{\bsA}{\boldsymbol{A}}
\newcommand{\bsH}{\boldsymbol{H}}
\newcommand{\bsI}{\boldsymbol{I}}
\newcommand{\bsP}{\boldsymbol{P}}
\newcommand{\bsR}{\boldsymbol{R}}
\newcommand{\bsS}{\boldsymbol{S}}
\newcommand{\bsX}{\boldsymbol{X}}
\newcommand{\bsY}{\boldsymbol{Y}}
\newcommand{\bsZ}{\boldsymbol{Z}}
\newcommand{\bse}{\boldsymbol{e}}
\newcommand{\bsq}{\boldsymbol{q}}
\newcommand{\bsy}{\boldsymbol{y}}
\newcommand{\bsbeta}{\boldsymbol{\beta}}
\newcommand{\fish}{\mathrm{F}}
\newcommand{\Fish}{\mathrm{F}}
\renewcommand{\phi}{\varphi}
\newcommand{\ind}{\mathds{1}}
\newcommand{\inds}[1]{\mathds{1}_{\{#1\}}}
\renewcommand{\to}{\rightarrow}
\newcommand{\sumin}{\sum\limits_{i=1}^n}
\newcommand{\ofbr}[1]{\bigl( \{ #1 \} \bigr)}     % Например, вероятность события. Большие круглые, нормальные фигурные скобки вокруг аргумента
\newcommand{\Ofbr}[1]{\Bigl( \bigl\{ #1 \bigr\} \Bigr)} % Например, вероятность события. Больше больших круглые, большие фигурные скобки вокруг аргумента
\newcommand{\oeq}{{}\textcircled{\raisebox{-0.4pt}{{}={}}}{}} % Равно в кружке
\newcommand{\og}{\textcircled{\raisebox{-0.4pt}{>}}}  % Знак больше в кружке

% вместо горизонтальной делаем косую черточку в нестрогих неравенствах
\renewcommand{\le}{\leqslant}
\renewcommand{\ge}{\geqslant}
\renewcommand{\leq}{\leqslant}
\renewcommand{\geq}{\geqslant}


\newcommand{\figb}[1]{\bigl\{ #1  \bigr\}} % большие фигурные скобки вокруг аргумента
\newcommand{\figB}[1]{\Bigl\{ #1  \Bigr\}} % Больше больших фигурные скобки вокруг аргумента
\newcommand{\parb}[1]{\bigl( #1  \bigr)}   % большие скобки вокруг аргумента
\newcommand{\parB}[1]{\Bigl( #1  \Bigr)}   % Больше больших круглые скобки вокруг аргумента
\newcommand{\parbb}[1]{\biggl( #1  \biggr)} % большие-большие круглые скобки вокруг аргумента
\newcommand{\br}[1]{\left( #1  \right)}    % круглые скобки, подгоняемые по размеру аргумента
\newcommand{\fbr}[1]{\left\{ #1  \right\}} % фигурные скобки, подгоняемые по размеру аргумента
\newcommand{\eqdef}{\mathrel{\stackrel{\rm def}=}} % знак равно по определению
\newcommand{\const}{\mathrm{const}}        % const прямым начертанием
\newcommand{\zdt}[1]{\textit{#1}}
\newcommand{\ENG}[1]{\foreignlanguage{british}{#1}}
\newcommand{\ENGs}{\selectlanguage{british}}
\newcommand{\RUSs}{\selectlanguage{russian}}
\newcommand{\iid}{\text{i.\hspace{1pt}i.\hspace{1pt}d.}}

\newdimen\theoremskip
\theoremskip=0pt
\newenvironment{note}{\par\vskip\theoremskip\textbf{Замечание.\xspace}}{\par\vskip\theoremskip}
\newenvironment{hint}{\par\vskip\theoremskip\textbf{Подсказка.\xspace}}{\par\vskip\theoremskip}
\newenvironment{ist}{\par\vskip\theoremskip Источник:\xspace}{\par\vskip\theoremskip}

\newcommand*{\tabvrulel}[1]{\multicolumn{1}{|c}{#1}}
\newcommand*{\tabvruler}[1]{\multicolumn{1}{c|}{#1}}

\newcommand{\II}{{\fontencoding{X2}\selectfont\CYRII}}   % I десятеричное (английская i неуместна)
\newcommand{\ii}{{\fontencoding{X2}\selectfont\cyrii}}   % i десятеричное
\newcommand{\EE}{{\fontencoding{X2}\selectfont\CYRYAT}}  % ЯТЬ
\newcommand{\ee}{{\fontencoding{X2}\selectfont\cyryat}}  % ять
\newcommand{\FF}{{\fontencoding{X2}\selectfont\CYROTLD}} % ФИТА
\newcommand{\ff}{{\fontencoding{X2}\selectfont\cyrotld}} % фита
\newcommand{\YY}{{\fontencoding{X2}\selectfont\CYRIZH}}  % ИЖИЦА
\newcommand{\yy}{{\fontencoding{X2}\selectfont\cyrizh}}  % ижица

%%%%%%%%%%%%%%%%%%%%% Определение разрядки разреженного текста и задание красивых многоточий
\sodef\so{}{.15em}{1em plus1em}{.3em plus.05em minus.05em}
\newcommand{\ldotst}{\so{...}}
\newcommand{\ldotsq}{\so{?\hbox{\hspace{-0.61ex}}..}}
\newcommand{\ldotse}{\so{!..}}
%%%%%%%%%%%%%%%%%%%%%%%%%%%%%%%%%%%%%%%%%%%%%%%%%%%%%%%%%%%%%%%%%%%%%%

%%%%%%%%%%%%%%%%%%%%%%%%%%%%% Команда для переноса символов бинарных операций
\def\hm#1{#1\nobreak\discretionary{}{\hbox{$#1$}}{}}
%%%%%%%%%%%%%%%%%%%%%%%%%%%%%%%%%%%%%%%%%%%%%%%%%%%%%%%%%%%%%%%%%%%%%%

%\setlist[enumerate,1]{label=\arabic*., ref=\arabic*, partopsep=0pt plus 2pt, topsep=0pt plus 1.5pt,itemsep=0pt plus .5pt,parsep=0pt plus .5pt}
%\setlist[itemize,1]{partopsep=0pt plus 2pt, topsep=0pt plus 1.5pt,itemsep=0pt plus .5pt,parsep=0pt plus .5pt}

% Эти парни затем, если вдруг не захочется управлять списками из-под уютненького enumitem
% или если будет жизненно важно, чтобы в списках были именно русские буквы.
%\setlength{\partopsep}{0pt plus 3pt}
%\setlength{\topsep}{0pt plus 2pt}
%\setlength{\itemsep}{0 plus 1pt}
%\setlength{\parsep}{0 plus 1pt}

%на всякий случай пока есть
%теоремы без нумерации и имени
%\newtheorem*{theor}{Теорема}

%"Определения","Замечания"
%и "Гипотезы" не нумеруются
%\newtheorem*{defin}{Определение}
%\newtheorem*{rem}{Замечание}
%\newtheorem*{conj}{Гипотеза}

%"Теоремы" и "Леммы" нумеруются
%по главам и согласованно м/у собой
%\newtheorem{theorem}{Теорема}
%\newtheorem{lemma}[theorem]{Лемма}

% Утверждения нумеруются по главам
% независимо от Лемм и Теорем
%\newtheorem{prop}{Утверждение}
%\newtheorem{cor}{Следствие}

\input{emetrix_preamble}




\usepackage[bibencoding = auto,
style = alphabetic,
backend = biber,
citestyle = alphabetic,
sorting = none]{biblatex}

\addbibresource{metrics_pro.bib}

\def \RR{\mathbb{R}}
\def \cN{\mathcal{N}}
\def \htheta{\hat{\theta}}

\title{Заметки к семинарам по эконометрике}
\author{Винни-Пух}
\date{\today}


% делаем короче интервал в списках
\setlength{\itemsep}{0pt}
\setlength{\parskip}{0pt}
\setlength{\parsep}{0pt}


\DeclareMathOperator{\Med}{Med}


\usepackage{answers} % дележка условий и ответов

%\newtheorem{problem}{Задача}
%\numberwithin{problem}{section}

\Newassociation{sol}{solution}{solution_file}
% sol — имя окружения внутри задач
% solution — имя окружения внутри solution_file
% solution_file — имя файла в который будет идти запись решений
% можно изменить далее по ходу
\Opensolutionfile{solution_file}[all_solutions]
% в квадратных скобках фактическое имя файла


% магия для автоматических гиперссылок задача-решение
\newlist{myenum}{enumerate}{3}
% \newcounter{problem}[chapter] % нумерация задач внутри глав
\newcounter{problem}[section]

\newenvironment{problem}%
{%
\refstepcounter{problem}%
%  hyperlink to solution
     \hypertarget{problem:{\thesection.\theproblem}}{} % нумерация внутри глав
     % \hypertarget{problem:{\theproblem}}{}
     \Writetofile{solution_file}{\protect\hypertarget{soln:\thesection.\theproblem}{}}
     %\Writetofile{solution_file}{\protect\hypertarget{soln:\theproblem}{}}
     \begin{myenum}[label=\bfseries\protect\hyperlink{soln:\thesection.\theproblem}{\thesection.\theproblem},ref=\thesection.\theproblem]
     % \begin{myenum}[label=\bfseries\protect\hyperlink{soln:\theproblem}{\theproblem},ref=\theproblem]
     \item%
    }%
    {%
    \end{myenum}}
% для гиперссылок обратно надо переопределять окружение
% это происходит непосредственно перед подключением файла с решениями





\begin{document}

% \maketitle % ставим сюда название, автора и время создания

\section{МНК — это\ldots}

Минитеория:

\begin{enumerate}
\item Истинная модель. Например, $y_i = \beta_1 + \beta_2 x_i + \beta_3 z_i + u_i$.
\item Формула для прогнозов. Например, $\hy_i = \hb_1 + \hb_2 x_i + \hb_3 z_i$.
\item Метод наименьших квадратов, $\sum (y_i - \hy_i)^2 \to \min$.
\end{enumerate}

Задачи:
\begin{problem}
Каждый день Маша ест конфеты и решает задачи по эконометрике. Пусть $x_i$ — количество решённых задач, а $y_i$ — количество съеденных конфет.

\begin{tabular}{cc}
\toprule
$x_i$ & $y_i$ \\
\midrule
1 & 1 \\
2 & 2 \\
2 & 4 \\
\bottomrule
\end{tabular}

\begin{enumerate}
\item Рассмотрим модель $y_i = \beta x_i + u_i$:
\begin{enumerate}
\item Найдите МНК-оценку $\hb$ для имеющихся трёх наблюдений.
\item Нарисуйте исходные точки и полученную прямую регрессии.
\item Выведите формулу для $\hb$ в общем виде для $n$ наблюдений.
\end{enumerate}

\item Рассмотрим модель $y_i = \beta_1 + \beta_2 x_i + u_i$:
\begin{enumerate}
\item Найдите МНК-оценки $\hb_1$ и $\hb_2$ для имеющихся трёх наблюдений.
\item Нарисуйте исходные точки и полученную прямую регрессии.
\item Выведите формулу для $\hb_2$ в общем виде для $n$ наблюдений.
\end{enumerate}
\end{enumerate}


\begin{sol}
\begin{enumerate}
\item
\begin{enumerate}
\item $\hb = 13/9$
\item[c)] $\hb = \frac{\sum_{i=1}^n x_i y_i}{\sum_{i=1}^n x_i^2}$
\end{enumerate}
\item
\begin{enumerate}
\item $\hb_1 = -1$, $\hb_2 = 2$
\item[c)] $\hb_2 = \frac{\sum_{i=1}^n (x_i - \bar x)(y_i - \bar y)}{\sum_{i=1}^n (x_i - \bar x)^2}$
\end{enumerate}
\end{enumerate}
\end{sol}
\end{problem}


\begin{problem}
Упростите выражения:
\begin{enumerate}
\item $n\bar x - \sum x_i$
\item $\sum (x_i - \bar x)\bar x$
\item $\sum (x_i - \bar x)\bar z$
\item $\sum (x_i - \bar x)^2 + n \bar{x}^2$
\end{enumerate}

\begin{sol}
\begin{enumerate}
\item $0$
\item $0$
\item $0$
\item $\sum x_i^2$
\end{enumerate}
\end{sol}
\end{problem}


\begin{problem}
При помощи метода наименьших квадратов найдите оценку неизвестного параметра $\theta$ в следующих моделях:

\begin{enumerate}
\item $y_i = \theta + \theta x_i + \varepsilon_i$;
\item $y_i = 1 + \theta x_i + \e_i$;
\item $y_i = \theta / x_i + \e_i$;
\item $y_i = \theta x_i + (1-\theta)z_i+\e_i$.
\end{enumerate}

\begin{sol}
\begin{enumerate}
\item $\htheta = \frac{\sum y_i (1 + x_i)}{\sum (1 + x_i)^2}$
\item $\htheta = \frac{\sum (y_i - 1) x_i}{\sum x_i^2}$
\item $\htheta = \frac{\sum (y_i / x_i^2)}{\sum (1 / x_i^3)}$
\item $\htheta = \sum \left((y_i - z_i)(x_i - z_i) \right) / \sum \left(x_i - z_i\right)^2$
\end{enumerate}
\end{sol}
\end{problem}

\begin{problem}
Найдите МНК-оценки параметров $\alpha$ и $\beta$ в модели $y_i = \alpha + \beta y_i + \e_i$.


\begin{sol}
\(\hat{\alpha} = 0, \ \hb = 1 \)
\end{sol}
\end{problem}


\begin{problem}
Рассмотрите модели $y_i = \alpha + \beta (y_i + z_i) + \e_i$, $z_i = \gamma + \delta(y_i+z_i) + \e_i$.
\begin{enumerate}
\item Как связаны между собой $\hat{\alpha}$ и $\hat{\gamma}$?
\item Как связаны между собой $\hb$ и $\hat{\delta}$?
\end{enumerate}


\begin{sol} % 1.5.
Рассмотрим регрессию суммы $(y_i + z_i)$ на саму себя. Естественно, в ней
\[
\widehat{y_i + z_i} = 0 + 1 \cdot (y_i + z_i).
\]

Отсюда получаем, что $\hat{\alpha} + \hat{\gamma} = 0$ и $\hb + \hat{\delta} = 1$.
\end{sol}
\end{problem}




\begin{problem}
Как связаны МНК-оценки параметров $\alpha, \beta$ и $\gamma, \delta$ в моделях $y_i = \alpha + \beta x_i + \e_i$ и $z_i = \gamma + \delta x_i + \upsilon_i$, если $z_i = 2 y_i$?


\begin{sol}

Исходя из условия, нужно оценить методом МНК коэффициенты двух следующих моделей:
\[y_i = \alpha + \beta x_i + \e_i \]
\[y_i = \frac{\gamma}{2} + \frac{\delta}{2} x_i + \frac{1}{2} v_i \]

Заметим, что на минимизацию суммы квадратов остатков коэффициент \(1/2\) не влияет, следовательно:
\[\hat{\gamma} = 2\hat{\alpha}, \ \hat{\delta} = 2 \hb  \]

\end{sol}
\end{problem}


\begin{problem}
Для модели $y_i = \beta_1 x_i + \beta_2 z_i + \e_i$ решите условную задачу о наименьших квадратах:
\[
Q(\beta_1, \beta_2) := \sum_{i=1}^n (y_i - \hb_1 x_i - \hb_2 z_i)^2 \rightarrow \underset{\hb_1 + \hb_2 = 1}{\min}.
\]


\begin{sol}
Выпишем задачу:
\[
\begin{cases}
RSS = \sum\limits_{i=1}^{n}(y_i - \hb_1x_i - \hb_2z_i)^2 \rightarrow \min\limits_{\hb_1, \hb_2}\\
\hb_1 + \hb_2 = 1
\end{cases}
\]

Можем превратить ее в задачу минимизации функции одного аргумента:
\[
RSS =  \sum\limits_{i=1}^{n}(y_i - x_i - \hb_2(z_i-x_i))^2 \rightarrow \min_{\hb_2}
\]

Выпишем условия первого порядка:
\[
\frac{\partial RSS}{\partial \hb_2} = \sum\limits_{i=1}^{n}2(y_i-x_i-\hb_2(z_i-x_i))(x_i-z_i)=0
\]

Отсюда:
\[
\sum\limits_{i=1}^{n}(y_i-x_i)(x_i-z_i) + \hb_2\sum\limits_{i=1}^{n}(z_i-x_i)^2 = 0 \Rightarrow \hb_2 = \frac{\sum\limits_{i=1}^n (y_i-x_i)(z_i-x_i)}{\sum\limits_{i=1}^n (z_i-x_i)^2}
\]

А $\hb_1$ найдется из соотношения $\hb_1+\hb_2 = 1$.

\end{sol}
\end{problem}

\begin{problem}
Перед нами два золотых слитка и весы, производящие взвешивания с ошибками. Взвесив первый слиток, мы получили результат $300$ грамм, взвесив второй слиток — $200$ грамм, взвесив оба слитка — $400$ грамм. Оцените вес каждого слитка методом наименьших квадратов.

\begin{sol}
Обозначив вес первого слитка за \(\beta_1\), вес второго слитка за \(\beta_2\), а показания весов за \(y_i\), получим, что
\[y_1 = \beta_1 + \e_1, \ y_2 = \beta_2 + \e_2, \ y_3 = \beta_1 + \beta_2 + \e_3\]

Тогда
\[(300 - \beta_1)^2 + (200 - \beta_2)^2 + (400 - \beta_1 - \beta_2)^2 \rightarrow \min \limits_{\beta_1,\  \beta_2} \]
\[\hb_1 = \frac{800}{3}, \ \hb_2 = \frac{500}{3} \]
\end{sol}
\end{problem}


\begin{problem}
Аня и Настя утверждают, что лектор опоздал на 10 минут. Таня считает, что лектор опоздал на 3 минуты. С помощью МНК оцените, на сколько опоздал лектор.

\begin{sol}
Можем воспользоваться готовой формулой для регрессии на константу:
\[
\hb = \bar{y} = \frac{10+10+3}{3} = \frac{23}{3}
\]

(можно решить задачу $2(10-\beta)^2 + (3-\beta)^2\rightarrow \min$)

\end{sol}
\end{problem}

\begin{problem}
Есть двести наблюдений. Вовочка оценил модель $\hy_i=\hb_1+\hb_2 x_i$ по первой сотне наблюдений.
Петечка оценил модель $\hy_i=\hat{\gamma}_1+\hat{\gamma}_2 x_i$ по второй сотне наблюдений.
Машенька оценила модель $\hy_i=\hat{\phi}_1+\hat{\phi}_2 x_i$ по всем наблюдениям.
\begin{enumerate}
\item Возможно ли, что $\hb_2>0$, $\hat{\gamma}_2>0$, но $\hat{\phi}_2<0$?
\item Возможно ли, что $\hb_1>0$, $\hat{\gamma}_1>0$, но $\hat{\phi}_1<0$?
\item Возможно ли одновременное выполнение всех упомянутых условий?
\item Возможно ли одновременное выполнение всех упомянутых условий,
если в каждой сотне наблюдений $\sum x_i > 0$?
\end{enumerate}

\begin{sol}
\begin{enumerate}
\item Да.
\item Да.
\item Да.
\item Нет. Из условия первого порядка для первой выборки следует,
что $\sum_A y_i = \hb_1 + \hb_2 \sum x_i$. Значит $\sum_A y_i > 0$. Аналогично, $\sum_B y_i >0$,
но $\sum y_i <0$.
\end{enumerate}
\end{sol}
\end{problem}


\begin{problem}

Эконометрист Вовочка собрал интересный набор данных по студентам третьего курса:
\begin{itemize}
\item переменная $y_i$ — количество пирожков, съеденных $i$-ым студентом за прошлый год;
\item переменная $f_i$, которая равна 1, если $i$-ый человек в выборке — женщина, и 0, если мужчина.
\item переменная\footnote{Это нетолерантная задача и здесь либо $f$ равно 1, либо $m$} $m_i$, которая равна 1, если $i$-ый человек в выборке — мужчина, и 0, если женщина.
\end{itemize}

Вовочка попробовал оценить 4 регрессии:
\begin{enumerate}
\item[A:] $y$ на константу и $f$, $\hy_i = \hat \alpha_1 + \hat \alpha_2 f_i$;
\item[B:] $y$ на константу и $m$, $\hy_i = \hb_1 + \hb_2 m_i$;
\item[C:] $y$ на $f$ и $m$ без константы, $\hy_i = \hat \gamma_1 f_i + \hat \gamma_2 m_i$;
\item[D:] $y$ на константу, $f$ и $m$, $\hy_i = \hat \delta_1 + \hat \delta_2 f_i + \hat \delta_3 m_i$;
\end{enumerate}

\begin{enumerate}
\item Какой смысл будут иметь оцениваемые коэффициенты?
\item Как связаны между собой оценки коэффициентов этих регрессий?
\end{enumerate}


\begin{sol}
\end{sol}
\end{problem}


\begin{problem}
Эконометрист Вовочка оценил методом наименьших квадратов модель 1, $y_i=\b_1+\b_2 x_i+\b_3 z_i+\e_i$,
а затем модель 2, $y_i=\b_1+\b_2 x_i+\b_3 z_i+\b_4 w_i+\e$.
Сравните полученные $ESS$, $RSS$, $TSS$ и $R^2$.

\begin{sol}
Поскольку значения $y$ остались теми же, $TSS_1 = TSS_2$.

Добавление ещё одного регрессора не уменьшит точность оценки, то есть
$RSS_2 \leq RSS_1$, $ESS_2 \geq ESS_1$.

Тогда и коэффициент детерминации $R^2 = ESS / TSS$ не уменьшится, то есть
$R^2_2 \geq R^2_1$.
\end{sol}
\end{problem}


\begin{problem}
Что происходит с $TSS$, $RSS$, $ESS$, $R^2$ при добавлении нового наблюдения? Если величина может изменяться только в одну сторону, то докажите это. Если возможны и рост, и падение, то приведите пример.


\begin{sol}
Пусть \(\bar{y}\) — средний \(y\) до добавления нового наблюдения, \(\bar{y}'\) — после добавления нового наблюдения. Будем считать, что изначально было \(n\) наблюдений. Заметим, что
\[\bar{y}' = \frac{(y_1 + \ldots + y_n) + y_{n+1}}{n + 1} = \frac{n \bar{y} + y_{n + 1}}{n + 1} = \frac{n}{n+ 1}\bar{y} + \frac{1}{n+1}y_{n+1}\]

Покажем, что \(TSS\) может только увеличится при добавлении нового наблюдения (остается неизменным при \(y_{n+1} = \bar{y}\)):
\[TSS'= \sum_{i = 1}^{n + 1} (y_i - \bar{y}')^2 = \sum_{i = 1}^{n} (y_i - \bar{y} + \bar{y} - \bar{y}')^2 + (y_{n + 1} - \bar{y}')^2 = \]
\[=\sum_{i = 1}^{n} (y_i - \bar{y})^2 + n(\bar{y} - \bar{y}')^2 + (y_{n + 1} - \bar{y}')^2  = TSS + \frac{n}{n+1} (y_{n+1} - \bar{y})^2\]

Следовательно, \(TSS' \geqslant TSS\).

Также сумма \(RSS\) может только вырасти или остаться постоянной при добавлении нового наблюдения. Действительно, новое $(n+1)$-ое слагаемое в сумме неотрицательно. А сумма $n$ слагаемых минимальна при старых коэффициентах, а не при новых.

\(ESS\) и \(R^2\) могут меняться в обе стороны. Например, рассмотрим ситуацию, где точки лежат симметрично относительно некоторой горизонтальной прямой. При этом $ESS=0$. Добавим наблюдение — $ESS$ вырастет, удалим наблюдение — $ESS$ вырастет.
\end{sol}
\end{problem}



\begin{problem}
Эконометресса Аглая подглядела, что у эконометрессы Жозефины получился $R^2$ равный $0.99$ по 300 наблюдениям. От чёрной зависти Аглая не может ни есть, ни спать.

\begin{enumerate}
\item Аглая добавила в набор данных Жозефины ещё 300 наблюдений с такими же регрессорами, но противоположными по знаку игреками, чем были у Жозефины. Как изменится $R^2$?
\item Жозефина заметила, что Аглая добавила 300 наблюдений и вычеркнула их, 
вернув набор данных в исходное состояние. Хитрая Аглая решила тогда добавить всего одно наблюдение так, чтобы $R^2$ упал до нуля. 
Удастся ли ей это сделать?
\end{enumerate}


\begin{sol}
\begin{enumerate}
\item $R^2$ упал до нуля.
\item Да, можно. Если добавить точку далеко слева внизу от исходного набора данных, то наклон линии регрессии будет положительный. Если далеко справа внизу, то отрицательный. Будем двигать точку так, чтобы поймать нулевой наклон прямой. Получим $ESS=0$.
\end{enumerate}
\end{sol}
\end{problem}


\begin{problem}
На работе Феофан построил парную регрессию по трём наблюдениям и посчитал прогнозы $\hat{y}_i$. Придя домой он отчасти вспомнил результаты:

\begin{tabular}{rr}
\toprule
$y_i$ & $\hy_i$ \\
\midrule
$0$ & $1$ \\
$6$ & ? \\
$6$ & ? \\
\bottomrule
\end{tabular}

Поднапрягшись, Феофан вспомнил, что третий прогноз был больше второго.
Помогите Феофану восстановить пропущенные значения.


\begin{sol}
На две неизвестных $a$ и $b$ нужно два уравнения. Эти два уравнения — ортогональность вектора остатков плоскости регрессоров. А именно:

\[
\begin{cases}
\sum_i (y_i - \hy_i) = 0 \\
\sum_i (y_i - \hy_i) \hy_i = 0 \\
\end{cases}
\]

В нашем случае

\[
\begin{cases}
-1 +(6-a) + (6-b) = 0 \\
-1 + (6 - a)a + (6-b)b = 0 \\
\end{cases}
\]

Решаем квадратное уравнение и получаем два решения: $a=4$ и $a=7$. Итого: $a=4$, $b=7$.
\end{sol}
\end{problem}


\begin{problem}
Вся выборка поделена на две части. Возможны ли такие ситуации:
\begin{enumerate}
    \item Выборочная корреляция между $y$ и $x$ примерно равна нулю в каждой части,
    а по всей выборке примерно равна единице;
    \item Выборочная корреляция между $y$ и $x$ примерно равна единице в каждой части,
    а по всей выборке примерно равна нулю?
\end{enumerate}

\begin{sol}
Обе ситуации возможны.
\end{sol}
\end{problem}


\begin{problem}
Бесстрашный исследователь Ипполит оценил парную регрессию. При этом оказалось,
каждый $x_i >0$ и обе оценки коэффициентов $\hb_1$ и $\hb_2$ также положительны.

\begin{enumerate}
  \item Возможно ли, что среди $\hat y_i$ есть отрицательные? Среди $y_i$ есть отрицательные?
  \item Возможно ли, что сумма $\sum \hat y_i$ отрицательна? Сумма $\sum y_i$ отрицательна?
  \item Как изменятся ответы, если известно, что $\sum x_i >0$?
\end{enumerate}

\begin{sol}
  \begin{enumerate}
    \item нет, да
    \item нет, нет
    \item да, да, нет, нет
  \end{enumerate}
\end{sol}
\end{problem}



\begin{problem}
Предложите способ подсчёта корреляции
\begin{enumerate}
  \item между бинарной переменной и факторной переменной;
  \item между количественной переменной и факторной переменной.
\end{enumerate}
  \begin{sol}
    Заменяем факторную переменную на набор бинарных предикторов. 
    С помощью этих предикторов прогнозируем бинарную или количественную переменную.
    Считаем $R^2$. Извлекаем корень. 
  \end{sol}
\end{problem}
  





\section{Дифференциал — просто няшка!}

Минитеория.

Дифференциал для матриц подчиняется правилам:

\begin{enumerate}
\item $d(A+B) = dA + dB$;
\item Если $A$ — матрица констант, то $dA = 0$;
\item $d(AB) = dA \cdot B + A \cdot dB$. Если $A$ — матрица констант, то $d(AB) = AdB$;
\end{enumerate}

Штрих у матрицы традиционно обозначает не производную, а транспонирование, $A'=A^T$.


\begin{problem}
    Вспомним дифференциал :)
    \begin{enumerate}
        \item Известно, что $f(x) = x^2 + 3x$. Найдите $f'(x)$ и $df$. Чему равен $df$ в точке $x=5$ при $dx=0.1$?
        \item Известно, что $f(x_1, x_2)=x_1^2 + 3x_1x_2^3$. Найдите $df$. Чему равен $df$ в точке $x_1=-2$, $x_2=1$ при $dx_1=0.1$ и $dx_2=-0.1$?
        \item Известно, что $F=\begin{pmatrix}
                5 & 6x_1 \\
                x_1x_2 & x_1^2x_2 \\
            \end{pmatrix}$. Найдите $dF$.
        \item Известно, что $F=\begin{pmatrix}
                7 & 8 & 9 \\
                2 & -1 & -2 \\
            \end{pmatrix}$. Найдите $dF$.
        \item Матрица $F$ имеет размер $2\times 2$, в строке $i$ столбце $j$ у неё находится элемент $f_{ij}$.
            Выпишите выражение $\tr(F'dF)$ в явном виде без матриц.
    \end{enumerate}
\begin{sol}
\begin{enumerate}
\item $f'(x) = 2x + 3$, $df = 2xdx + 3dx$, $df = 1.3$
\item $df = 2 x_1 d x_1 + 3 d x_1 \cdot x_2^3 + 3x_1 \cdot 3 x_2^2 dx_2$, $df = 1.7$
\end{enumerate}
\end{sol}
\end{problem}


\begin{problem}
Пусть $A$, $B$ — матрицы констант, $R$ — матрица переменных, $r$ — вектор столбец переменных. 
Применив базовые правила дифференцирования найдите:

\begin{enumerate}
\item $d(ARB)$;
\item $d(r'r)$;
\item $d(r'Ar)$;
\item $d(R^{-1})$, воспользовавшись тем, что $R^{-1} \cdot R = I$;
\item $d \cos(r'r)$;
\item $d(r'Ar/r'r)$.
\end{enumerate}

Упростите ответ для случая симметричной матрицы $A$.

\begin{sol}
\begin{enumerate}
\item $A(dR)B$
\item $2r'dr$
\item $r'(A'+A)dr$
\item $R^{-1}\cdot dR \cdot R^{-1}$
\item $-\sin(r'r)\cdot 2r'dr$
\item $\frac{r'(A'+A)dr \cdot r'r - r'Ar2r'dr}{(r'r)^2}$
\end{enumerate}
\end{sol}
\end{problem}


\begin{problem}
В методе наименьших квадратов минимизируется функция
\[
Q(\hb) = (y - X\hb)'(y - X\hb).
\]

\begin{enumerate}
\item Найдите $dQ(\hb)$ и $d^2Q(\hb)$;
\item Выпишите условия первого порядка для задачи МНК;
\item Выразите $\hb$ предполагая, что $X'X$ обратима.
\end{enumerate}


\begin{sol}
\begin{enumerate}
\item $dQ(\hb) = 2(y-X\hb)^T (-X) d\hb$, $d^2Q(\hb) = 2d\hb^T X^T X d\hb$
\item $dQ(\hb) = 0$
\item $\hb = (X^T X)^{-1} X^T y$
\end{enumerate}
\end{sol}
\end{problem}

\begin{problem}
В гребневой регрессии (ridge regression) минимизируется функция
\[
Q(\hb) = (y - X\hb)'(y - X\hb) + \lambda \hb' \hb,
\]
где $\lambda$ — положительный параметр, штрафующий функцию за слишком большие значения $\hb$.

\begin{enumerate}
\item Найдите $dQ(\hb)$ и $d^2Q(\hb)$;
\item Выпишите условия первого порядка для задачи LASSO;
\item Выразите $\hb$.
\end{enumerate}

\begin{sol}
\begin{enumerate}
\item $dQ(\hb) = -2((y-X\hb)^T X + \lambda \hb^T) d\hb$, $d^2Q(\hb)=2d\hb^T(X^T X - \lambda I) d\hb$
\item $dQ(\hb) = 0$
\item $\hb = (X^T X - \lambda I)^{-1} X^T y$
\end{enumerate}
\end{sol}
\end{problem}



\begin{problem}
Исследователь Никодим поймал 100 морских ежей и у каждого измерил длину, $a_i$,
и вес $b_i$. Вектор измерений, относящихся к одному ежу обозначим $y_i = \begin{pmatrix}
a_i \\
b_i \\
\end{pmatrix}$. Никодим считает, что ежи независимы друг от друга,
а длина и вес имеют совместное нормальное распределение
\[
y_i = \begin{pmatrix}
a_i \\
b_i \\
\end{pmatrix} \sim \cN \left( \mu, C  \right)
\]

\begin{enumerate}
\item Выпишите логарифмическую функцию правдоподобия, $\ell(\mu, C)$;
\item Предполагая ковариационную матрицу известной, $C = \begin{pmatrix}
9 & 4 \\
4 & 6 \\
\end{pmatrix}$,
найдите $d \ell$ и оценку $\hat \mu$ методом максимального правдоподобия.

\item Предполагая, вектор ожиданий известным, $\mu = \begin{pmatrix}
10 \\
5 \\
\end{pmatrix}$,
 найдите $d \ell$ и оценку $\hat C$ методом максимального правдоподобия.


\item Найдите $d \ell(\mu, C)$ и оценки для параметров $\mu$ и $C$,
в случае, когда $\mu$ и $C$ неизвестны.
\end{enumerate}


\begin{sol}
\begin{enumerate}
\item $\hat \mu  = \sum y_i / n$
\item
\item
\end{enumerate}
\end{sol}
\end{problem}




\section{МНК в матрицах и геометрия!}


\begin{problem}
Рассмотрим регрессию $\hy_i = \hb_1 z_i + \hb_2 x_i$.
Все исходные данные поместим в матрицу $X$ и вектор $y$:
\[
X = \begin{pmatrix}
z_1 & x_1 \\
\vdots & \vdots \\
z_n & x_n \\
\end{pmatrix} \;
y = \begin{pmatrix}
y_1 \\
\vdots \\
y_n \\
\end{pmatrix}
\]
\begin{enumerate}
  \item Выпишите явно матрицы $X'$, $X'y$, $X'X$, $y'X$, $y'z$ и укажите их размер.
  \item Выпишите условия первого порядка для оценок $\hb_1$ и $\hb_2$ по методу наименьших квадратов.
  \item Запишите эти же условия в виде линейной системы
  \[
\begin{cases}
\hb_1 \cdot \ldots + \hb_2 \cdot \ldots = \ldots \\
\hb_1 \cdot \ldots + \hb_2 \cdot \ldots = \ldots \\
\end{cases}
  \]
  \item Как упростится данная система для регресии $\hy_i = \hb_1 + \hb_2 x_i$?
  \item Запишите систему условий первого порядка с помощью матрицы $X$ и вектора $y$;
\end{enumerate}

\begin{sol}
\[
\begin{cases}
\hb_1 \sum z_i^2 + \hb_2 \sum x_i z_i = \sum z_i y_ i \\
\hb_1 \sum x_i z_i + \hb_2 \sum x_i^2 = \sum x_i y_ i \\
\end{cases}
\]

\[
\begin{cases}
\hb_1 n + \hb_2 \sum x_i = \sum y_ i \\
\hb_1 \sum x_i + \hb_2 \sum x_i^2 = \sum x_i y_ i \\
\end{cases}
\]

\[
X'X\hb = X'y
\]

\end{sol}
\end{problem}



% 4.13
\begin{problem}
Пусть $y_i = \beta_1 + \beta_2 x_{i} + \beta_3 z_{i3} + \e_i$ — регрессионная модель, 
где $X = \begin{pmatrix} 1 & 0 & 0 \\ 1 & 0 & 0 \\ 1 & 0 & 0 \\ 1 & 1 & 0 \\ 1 & 1 & 1 \end{pmatrix}$, 
$y = \begin{pmatrix} 1 \\ 2 \\ 3 \\ 4 \\ 5 \end{pmatrix}$, 
$\beta = \begin{pmatrix} \beta_1 \\ \beta_2 \\ \beta_3 \end{pmatrix}$, 
$\e = \begin{pmatrix} \e_1 \\ \e_2 \\ \e_3 \\ \e_4 \\ \e_5  \end{pmatrix}$, ошибки 
$\e_i$ независимы и нормально распределены с $\E(\e)$ = 0, $Var(\e) = \sigma^2 I$. 

Для удобства расчётов даны матрицы: $X'X = \begin{pmatrix} 5 & 2 & 1 \\ 2 & 2 & 1\\ 1 & 1 & 1 \end{pmatrix}$ и 
$(X'X)^{-1}= \begin{pmatrix} 1/3 & -1/3 & 0 \\ -1/3 & 4/3 & -1 \\ 0 & -1 & 2 \end{pmatrix}$


\begin{enumerate}
\item Укажите число наблюдений.
\item Укажите число регрессоров в модели, учитывая свободный член.
\item Найдите $TSS = \sum_{i=1}^n (y_i - \bar y)^2$.
\item Методом МНК найдите оценку для вектора неизвестных коэффициентов.
\item Найдите вектор прогнозов $\hy$.
\item Найдите $RSS = \sum_{i=1}^n (y_i - \hy_i)^2$.
\item Чему равен $R^2$ в модели? Прокомментируйте полученное значение с точки зрения качества оценённого уравнения регрессии.\end{enumerate}
\begin{sol}

\begin{enumerate}
\item $n = 5$
\item $k = 3$
\item $TSS = 10$
\item $\hb = \begin{pmatrix} \hb_1 \\ \hb_2 \\ \hb_3 \end{pmatrix} = (X'X)^{-1}X'y = \begin{pmatrix} 2 \\ 2 \\ 1 \end{pmatrix}$
\item $\hy = X\hb$
\item $RSS = 2$
\item $R^2 = 1 - \frac {RSS}{TSS} = 0.8.$ $R^2$ высокий, построенная эконометрическая модель хорошо описывает данные
\end{enumerate}
\end{sol}
\end{problem}



\begin{problem}
Найдите на картинке все перпендикулярные векторы. Найдите на картинке все прямоугольные треугольники. Сформулируйте для них теоремы Пифагора.

\tdplotsetmaincoords{70}{110}
\begin{tikzpicture}[tdplot_main_coords]
\coordinate (hY) at (0,2.7,0);
\coordinate (Y) at (-2,2,2);
\coordinate (bY) at (-2,1,0);
\draw[thick,dotted, ->] (0,0,0) -- (-4,2,0) node[anchor=west]{$\vec{1}$};
\draw[thick,->] (0,0,0) -- (hY) node[anchor=west]{$\hy$};
\draw[thick,->] (0,0,0) -- (Y) node[anchor=south]{$y$};
\draw[thick,->] (0,0,0) -- (1,2,0) node[anchor=north]{$x$};
\draw[dotted] (hY) -- (Y);
\draw[dotted] (hY) -- (bY);
\draw[dotted] (Y) -- (bY);
\draw[thick,->] (0,0,0) -- (bY) node[anchor=south]{$\bar{y}\cdot \vec{1}$};
\end{tikzpicture}





\begin{sol}
$\sum y_i^2=\sum \hy_i^2+\sum \he_i^2$, $TSS=ESS+RSS$,
\end{sol}
\end{problem}



\begin{problem}
Покажите на картинке TSS, ESS, RSS, $R^2$, $\sCorr(\hy,y)$, $\sCov(\hy,y)$

\tdplotsetmaincoords{70}{110}
\begin{tikzpicture}[tdplot_main_coords]
\coordinate (hY) at (0,2.7,0);
\coordinate (Y) at (-2,2,2);
\coordinate (bY) at (-2,1,0);
\draw[thick,dotted, ->] (0,0,0) -- (-4,2,0) node[anchor=west]{$\vec{1}$};
\draw[thick,->] (0,0,0) -- (hY) node[anchor=west]{$\hy$};
\draw[thick,->] (0,0,0) -- (Y) node[anchor=south]{$y$};
\draw[thick,->] (0,0,0) -- (1,2,0) node[anchor=north]{$x$};
\draw[dotted] (hY) -- (Y);
\draw[dotted] (hY) -- (bY);
\draw[dotted] (Y) -- (bY);
\draw[thick,->] (0,0,0) -- (bY) node[anchor=south]{$\bar{y}\cdot \vec{1}$};
\end{tikzpicture}




\begin{sol}
$\sCorr(\hy, y)=\frac{\sCov(\hy, y)}{\sqrt{\sVar(\hy)\sVar{(y)}}}$

$\sCorr(\hy, y)^2=\frac{(\sCov(\hy, y))^2}{\sVar(\hy)\sVar{(y)}} $

$R^2\cdot TSS/(n-1)\cdot ESS/(n-1)=(\sCov(\hy, y))^2=(\sCov(\hy-\bar y, y-\bar y))^2$
Отсюда можно понять, что ковариация для двухмерного случая равна произведению длин векторов $\hy-\bar y$ и $y-\bar y$ — $\sqrt{ESS}$ и $\sqrt{TSS}$ на косинус угла между ними ($\sqrt{R^2}$). Геометрически скалярное произведение можно изобразить как произведение длин одного из векторов на проекцию второго вектора на первый. Если будет проецировать $y-\bar y\v1$ на $\hy-\bar y\v1$, то получим как раз $ESS$ — тот квадрат на рисунке, что уже построен.


$\sCov(\hy, y)=\sqrt{ESS^2/(n-1)^2}=ESS/(n-1)$


\end{sol}
\end{problem}


\begin{problem}
Предложите аналог $R^2$ для случая, когда константа среди регрессоров отсутствует. Аналог должен быть всегда в диапазоне $[0;1]$, совпадать с обычным $R^2$, когда среди регрессоров есть константа, равняться единице в случае нулевого $\he$.


\begin{sol}
Спроецируем единичный столбец на «плоскость», обозначим его $1'$. Делаем проекцию $y$ на «плоскость» и на $1'$. Далее аналогично.
\end{sol}
\end{problem}



\begin{problem}
Вася оценил регрессию $y$ на константу, $x$ и $z$. А затем, делать ему нечего, регрессию $y$ на константу и полученный $\hy$. Какие оценки коэффициентов у него получатся? Чему будет равна оценка дисперсии коэффицента при $\hy$? Почему оценка коэффициента неслучайна, а оценка её дисперсии положительна?


\begin{sol}
Проекция $y$ на $\hy$ это $\hy$, поэтому оценки коэффициентов будут 0 и 1. Оценка дисперсии $\frac{RSS}{(n-2)ESS}$. Нарушены предпосылки теоремы Гаусса-Маркова, например, ошибки новой модели в сумме дают 0, значит коррелированы.
\end{sol}
\end{problem}




\begin{problem}
При каких условиях $TSS=ESS+RSS$?

\begin{sol}
Либо в регрессию включена константа, либо единичный столбец (тут была опечатка, столбей) можно получить как линейную комбинацию регрессоров, например, включены дамми-переменные для каждого возможного значения качественной переменной.
\end{sol}
\end{problem}


\begin{problem}
Вася построил парную регрессию $y$ на $x$ и получил коэффициент наклона $1.4$. Построил парную регрессию $x$ на $y$ и получил коэффициент наклона $0.6$. Известно, что $y=x+z$.
\begin{enumerate}
    \item Найдите выборочные корреляции между $x$ и $y$, $y$ и $z$, $x$ и $z$;
    \item В какой пропорции соотносятся выборочные дисперсии $x$, $y$ и $z$?
\end{enumerate}
\begin{sol}
Для удобства центрируем мысленно все переменные. Это не меняет ни корреляций,
ни выборочных дисперсий, ни угловых коэффициентов в регрессиях.
В регрессиях при этом оценка коэффициента при константе превращается в ноль, но какое нам до неё дело? :)
Зато при нулевом среднем выборочные корреляции превратились в косинус угла между векторами:
\[
\sCorr(x,y) = \frac{\sum x_i y_i}{ \sqrt{\sum x_i^2 \sum y_i^2}} = \cos(x, y)
\]
И при нулевом среднем выборочная дисперсия — это длина вектора с точностью до умножения на $(n-1)$:
\[
\sVar(x) = \frac{\sum x_i^2}{n-1} = \frac{||x||^2}{n-1}
\]

Начать можно с геометрического смысла оценок МНК:

\[
\begin{cases}
1.4 = \frac{||y||}{||x||}\cos(x, y) \\
0.6 = \frac{||x||}{||y||}\cos(x, y) \\
\end{cases}
\]

Отсюда находим $||y||/||x|| = \hat{\sigma}_y / \hat{\sigma}_x$ и $\cos(x,y)=\sCorr(x,y)$

Дальше номер решается, например, по теореме косинусов.


\begin{enumerate}
\item $\sCorr(x,y) = \sqrt{0.84}$, $\sCorr(y,z) = \frac{\sqrt{70}}{10}$,
$\sCorr(x,z) = -\frac{\sqrt{30}}{10}$
\item $\frac{\hat{\sigma}_x^2}{\hat{\sigma}_y^2} = \frac{3}{7}$,
$\frac{\hat{\sigma}_z^2}{\hat{\sigma}_y^2} = \frac{8}{35}$,
$\frac{\hat{\sigma}_x^2}{\hat{\sigma}_z^2} = \frac{15}{8}$
\end{enumerate}
\end{sol}
\end{problem}


\begin{problem}
Какие матрицы являются положительно полуопределёнными?

\begin{enumerate}
  \item $X'X$;
  \item $XX'$;
  \item $H = X(X'X)^{-1}X'$;
  \item $I - H$;
  \item $A'(I-H)A$;
  \item $A'A - G(G'A^{-1}(A')^{-1}G)^{-1}G'$
\end{enumerate}
  \begin{sol}
    все :)
  \end{sol}
\end{problem}

\section{Распределения, связанные с проецированием}


\begin{problem}
Рассмотрим пространство $\RR^3$ и два подпространства в нём, $W = \left\{ (x_1, x_2, x_3) \mid x_1 + 2x_2 + x_3 =0 \right\}$ и $V = \Lin((1,2,3)^T)$.

\begin{enumerate}
\item Найдите $\dim V$, $\dim W$, $\dim V \cap W$, $\dim V^{\perp}$, $\dim W^{\perp}$.
\item Найдите проекцию произвольного вектора $u$ на $V$, $W$, $V\cap W$, $V^{\perp}$,
$W^{\perp}$. Найдите квадрат длины каждой проекции.
\item Как распределён квадрат длины проекции в каждом случае,
если дополнительно известно,
что вектор $u$ имеет многомерное стандартное нормальное распределение?
\end{enumerate}

\begin{sol}
$\dim V = 1$, $\dim W = 2$, $\dim V \cap W =0$, $\dim V^{\perp}=2$, $\dim W^{\perp}=1$.
Эти же числа и будут степенями свободы хи-квадрат распределения.
\end{sol}
\end{problem}





\begin{problem}
Рассмотрим пространство $\RR^n$, где $n>2$, и два подпространства в нём, $V = \Lin((1,1, \ldots, 1)^T)$
и $W = \left\{ x \mid x_1 = x_2 + x_3  + \ldots + x_n \right\}$.

\begin{enumerate}
\item Найдите $\dim V$, $\dim W$, $\dim V \cap W$, $\dim V^{\perp}$,
$\dim W^{\perp}$, $\dim V \cap W^{\perp}$, $\dim V^{\perp} \cap W$.

\item Найдите проекцию произвольного вектора $u$ на каждое упомянутое подпространство.
Найдите квадрат длины каждой проекции.
\item Как распределён квадрат длины проекции в каждом случае,
если дополнительно известно,
что вектор $u$ имеет многомерное стандартное нормальное распределение?
\end{enumerate}

\begin{sol}
$\dim V = 1$, $\dim W = n-1$, $\dim V \cap W =0$, $\dim V^{\perp}=n-1$, $\dim W^{\perp}=1$.
\end{sol}
\end{problem}

\begin{problem}
Храбрая исследовательница Евлампия оценивает модель
множественной регрессии $\hy = X\hb$.
Однако на самом деле $\beta_0$, и $y=u$,
где $u_i$ независимы и нормальны $u_i \sim \cN(0;\sigma^2)$.

Какое распределение в регрессии Евлампии имеют $\bar y$, $\sum y_i^2/\sigma^2$,
$\sum \hat y_i^2/\sigma^2$,
$n\bar y^2/\sigma^2$, $TSS/\sigma^2$, $RSS/\sigma^2$, $ESS/\sigma^2$?
\begin{sol}

\end{sol}
\end{problem}



\begin{problem}
Компоненты вектора $x=(x_1, x_2)'$ независимы и имеют стандартное нормальное распределение. Вектор $y$ задан формулой $y = (2x_1 + x_2 + 2, x_1 - x_2 - 1)$.
\begin{enumerate}
  \item Выпишите совместную функцию плотности вектора $x$;
  \item Нарисуйте на плоскости линии уровня функции плотности вектора $x$;
  \item Выпишите совместную функцию плотности вектора $y$;
  \item Найдите собственные векторы и собственные числа ковариационной матрицы вектора $y$;
  \item Нарисуйте на плоскости линии уровня функции плотности вектора $y$.
\end{enumerate}
\begin{sol}
\end{sol}
\end{problem}


\begin{problem}
Компоненты вектора $x=(x_1, x_2, x_3)'$ независимы и имеют стандартное нормальное распределение.

\begin{enumerate}
\item Как выглядят в пространстве поверхности уровня совместной функции плотности?
\item Рассмотрим три апельсина с кожурой одинаковой очень маленькой толщины: бэби-апельсин радиуса $0.1$, стандартный апельсин радиуса $1$ и гранд-апельсин радиуса $10$. В кожуру какого апельсина вектор $x$ попадает с наибольшей вероятностью?
\item Мы проецируем случайный вектор на $x$ на плоскость $2x_1 + 3x_2 - 7x_3 = 0$. Какое распределение имеет квадрат длины проекции?
\item Введём вектор $y$ независимый от $x$ и имеющий такое же распределение. Спроецируем вектор $x$ на плоскость проходящую через начало координат и перпендикулярную вектору $y$.  Какое распределение имеет квадрат длины проекции?
\end{enumerate}


\begin{sol}
Сферы с центром в начале координат. Проекция имеет хи-квадрат распределение с тремя степенями свободы.
Для нахождения максимальной вероятности максимизируем функцию
\[
\exp(-R^2/2) \cdot ((R+t)^3 - R^3) \to \max_R
\],
где $R$ — радиус мякоти, а $t$ — толщина кожуры апельсина. Оставляем только линейную часть по $t$ и затем максимизируем.

Наибольшая вероятность попасть в апельсин радиуса $R=1$.
\end{sol}
\end{problem}


\section{Ожидания и ковариационные матрицы}


\begin{problem}
Исследовательница Мишель собрала данные по 20 студентам. 
Переменная $y_i$ — количество решённых задач по эконометрике $i$-ым студентом, 
а $x_i$ — количество просмотренных серий любимого сериала за прошедший год. 
Оказалось, что $\sum y_i = 10$, $\sum x_i = 0$, $\sum x_i^2 = 40$, $\sum y_i^2 = 50$, $\sum x_i y_i = 60$.

\begin{enumerate}
\item Найдите МНК оценки коэффициентов парной регрессии.

\item В рамках предположения $\E(u_i|X) = 0$ найдите $\E(y_i|X)$, $\E(\hb_j|X)$, $\E(\hat u_i|X)$, $\E(\hat y_i|X)$.

\item Предположим дополнительно, что $\Var(u_i|X)=\sigma^2$ и $u_i$ при фиксированных $X$ независимы. Найдите $\Var(y_i|X)$, $\Var(y_i (x_i - \bar x)|X)$, $\Var(\sum y_i (x_i - \bar x)|X)$, $\Var(\hb_2 | X)$.

\end{enumerate}

\begin{sol}
\end{sol}
\end{problem}

\begin{problem}
Рассмотрим классическую линейную модель $y=X\beta + u$ с предпосылками Гаусса-Маркова: $\E(u|X) = 0$, $\Var(u|X) = \sigma^2 I$.

Для всех случайных векторов ($y$, $\hy$, $\hb$, $u$, $\hat u$, $\bar y$) найдите $\E(\cdot)$, $\Var(\cdot)$, $\Cov(\cdot, \cdot)$.

\begin{sol}
\end{sol}
\end{problem}



\begin{problem}
Пусть регрессионная модель $y_i = \beta_1 + \beta_2 x_{i2} + \beta_3 x_{i3} + \e_i$, $i = 1, \ldots, n$, задана в матричном виде при помощи уравнения $y = X \beta + \e$, где $\beta =  \begin{pmatrix}
\beta_1 & \beta_2 & \beta_3\\
\end{pmatrix} '$. Известно, что $\E \e = 0$ и $\Var (\e) = 4 \cdot I$. Известно также, что:

$y =  \begin{pmatrix}
1 \\
2 \\
3 \\
4 \\
5 \\
\end{pmatrix} $, $X =  \begin{pmatrix}
1 & 0 & 0 \\
1 & 0 & 0 \\
1 & 1 & 0 \\
1 & 1 & 0 \\
1 & 1 & 1 \\
\end{pmatrix} $

Для удобства расчётов ниже приведены матрицы:

$X' X =  \begin{pmatrix}
5 & 3 & 1 \\
3 & 3 & 1 \\
1 & 1 & 1 \\
\end{pmatrix} $ и $(X' X)^{-1} =  \begin{pmatrix}
0.5 & -0.5 & 0 \\
-0.5 & 1 & -0.5 \\
0 & -0.5 & 1.5 \\
\end{pmatrix} $.


\begin{enumerate}
\item Найдите $\E (\hs^2)$, $\hs^2$.
\item Найдите $\Var (\e_1)$, $\Var (\beta_1)$, $\Var (\hb_1)$, $\hVar(\hb_1)$, $\E (\hb_1^2) - \beta_1^2$;
\item Предполагая нормальность ошибок, постройте $95\%$ доверительный интервал для $\beta_2$.
\item Предполагая нормальность ошибок, проверьте гипотезу $H_0$: $\beta_2 = 0$;
\item Найдите $\Cov (\hb_2, \hb_3)$, $\hCov(\hb_2, \hb_3)$, $\Var (\hb_2 - \hb_3)$, $\hVar(\hb_2 - \hb_3)$;
\item Найдите $\Var (\beta_2 - \beta_3)$, $\Corr (\hb_2, \hb_3)$, $\hCorr(\hb_2, \hb_3)$;
\item Предполагая нормальность ошибок, проверьте гипотезу $H_0$: $\beta_2 = \beta_3$;
\end{enumerate}


\begin{sol}
\begin{enumerate}
\item $\Var(\e_1)=\Var(\e)_{(1,1)}=4\cdot I_{(1,1)}=4$
\item $\Var(\beta_1)=0$, так как $\beta_1$ — детерминированная величина.
\item $\Var(\hb_1)=\sigma^2(X'X)^{-1}_{(1,1)}=0.5\sigma^2=0.5\cdot 4=2$
\item $\hVar(\hb_1)=\hat\sigma^2(X'X)^{-1}_{(1,1)}=0.5\hat\sigma^2_{(1,1)}=0.5\frac{RSS}{5-3}=0.25RSS=0.25y'(I-X(X'X)^{-1}X')y=0.25\cdot 1=0.25$

$\hat\sigma^2=\frac{RSS}{n-k}=\frac12$.

\item Так как оценки МНК являются несмещёнными, то $\E(\hb)=\beta$, значит:
\[
\E(\hb_1)-\beta_1^2=\E(\hb_1)-(\E(\hb_1))^2=\hVar(\hb_1)=0.25
\]

\item $\Cov(\hb_2,\hb_3)=\sigma^2(X'X)^{-1}_{(2,3)}=4\cdot\left(-\frac12\right)=-2$
\item $\hCov(\hb_2,\hb_3)=\hVar(\hat\beta)_{(2,3)}=\hat\sigma^2(X'X)^{-1}_{(2,3)}=\frac{1}{2}\cdot\left(-\frac12\right)=-\frac14$

\item $\Var(\hb_2-\hb_3)=\Var(\hb_2)+\Var(\hb_3)+2\Cov(\hb_2,\hb_3)=\sigma^2((X'X)^{-1}_{(2,2)}+(X'X)^{-1}_{(3,3)}+2(X'X)^{-1}_{(2,3)}=4(1+1.5+2\cdot(-0.5))=6$

\item $\hVar(\hb_2-\hb_3)=\hVar(\hb_2)+\hVar(\hb_3)+2\hCov(\hb_2,\hb_3)=\hat\sigma^2((X'X)^{-1}_{(2,2)}+(X'X)^{-1}_{(3,3)}+2(X'X)^{-1}_{(2,3)}=\frac{1}{2}\cdot1.5=0.75$

\item $\Var(\beta_2-\beta_3)=0$

\item $\corr(\hb_2,\hb_3)=\frac{\Cov(\hb_2,\hb_3)}{\sqrt{\Var(\hb_2)\Var(\hb_3)}}=\frac{-2}{\sqrt{4\cdot6}}=-\frac{\sqrt6}{6}$

\item $\hCorr(\beta_2,\beta_3)=\frac{\hCov(\hb_2,\hb_3)}{\sqrt{\hVar(\hb_2)\hVar(\hb_3)}}=\frac{-\frac14}{\sqrt{\frac12\cdot\frac34}}=-\frac{\sqrt6}{6}$

\item $(n-k)\frac{\hat\sigma^2}{\sigma^2}\sim\chi^2_{n-k}$.
\[
\E\left((n-k)\frac{\hat\sigma^2}{\sigma^2}\right)=n-k
\]
\[
\E\left(\frac{\hat\sigma^2}{2}\right)=1
\]
\[
\E(\hat\sigma^2)=2
\]

\item $\hat\sigma^2=\frac{RSS}{n-k}=\frac12$

\end{enumerate}

\end{sol}
\end{problem}





\section{Гипотезы и интервалы}







\section{Гамма, бета}

\begin{problem}
Вася делает эксперименты без устали со скоростью $d$ экспериментов в минуту. Каждый эксперимент независимо от других может окончится успехом с вероятностью $p$ или неудачей.

Пусть $X$ — количество успехов за первую минуту, а $Y$ — номер опыта, в котором произошёл первый успех, $Z$ — время, когда случился первый успех.
\begin{enumerate}
\item Найдите $\P(X = k)$, $\E(X)$, $\Var(X)$. Как называется закон распределения $X$?
\item Найдите $\P(Y = k)$, $\E(Y)$, $\Var(Y)$. Как называется закон распределения $Y$?
\item Найдите $\P(Z \leq t)$, $\E(Z)$, $\Var(Z)$.
\end{enumerate}

Теперь Вася ускоряется и устремляет $d$ в бесконечность. Из-за того, что он торопится, $p$ начинает стремится к нулю :) Причем ожидаемое количество успехов за минуту оказывается постоянно и равно $\lambda$.

\begin{enumerate}[resume]
\item Выразите $p$ через $\lambda$ и $d$.
\item Найдите предел $\P(Z \leq t)$. Является ли предельная функция $\P(Z \leq t)$ непрерывной? Какая в предельном случае получается функция плотности у величины $Z$? Как называется этот закон распределения $Z$? Чему равен предел $\E(Z)$ и $\Var(Z)$?
\item Найдите предел вероятности $\P(X = k)$ и пределы $\E(X)$ и $\Var(X)$. Как называется предельный закон распределения $X$?
\end{enumerate}
\begin{sol}
\end{sol}
\end{problem}

\begin{problem}
    Энтомолог Джон Поллак ловит бабочек. На поимку $i$-ой бабочки у него уходит $Y_i$ минут, величины $Y_i$ независимы. Каждая $Y_i$ имеет экспоненциальное распределение с интенсивностью $\lambda$ бабочек в минуту. Всего он решил поймать $n$ бабочек. Рассмотрим величины $S=Y_1 + \ldots + Y_n$, $X_1 = Y_1 / S$, $X_2 = Y_2/S$, \ldots, $X_{n-1}=Y_{n-1}/S$.

\begin{enumerate}
    \item Выпишите совместную функцию плотности $Y_1$, \ldots, $Y_n$;
    \item Найдите совместную функцию плотности $X_1$, $X_2$, $X_3$, \ldots, $X_{n-1}$, $S$.
    \item Зависит ли величина $S$ и вектор $X_1$, $X_2$, \ldots, $X_{n-1}$?
    \item С точностью до сомножителя выпишите функцию плотности $S$. Как называется закон распределения $S$?
    \item С точностью до сомножителя выпишите совместную функцию плотности для $X_1$, \ldots, $X_{n-1}$.
\end{enumerate}

Рассмотрим также величины $Z_1 = Y_1 / (Y_1 + Y_2)$, $Z_2 = (Y_1 + Y_2) / (Y_1 + Y_2 + Y_3)$, \ldots, $Z_{n-1} = (Y_1 + \ldots + Y_{n-1}) / (Y_1 + \ldots + Y_n)$.

\begin{enumerate}[resume]
    \item Найдите совместную функцию плотности $Z_1$, $Z_2$, \ldots, $Z_{n-1}$, $S$.
    \item Зависимы ли величины $Z_1$, $Z_2$, \ldots, $Z_{n-1}$, $S$?
    \item С точностью до константы найдите частную функцию плотности $S$ и каждого $Z_i$ в отдельности;
\end{enumerate}


\begin{sol}
\end{sol}
\end{problem}





\begin{problem}
    Быстрый исследователь Вася снова проводит независимые идентичные опыты с очень высокой скоростью. В среднем $\lambda$ опытов в минуту оказываются успешными. Поэтому время до очередного успеха можно считать экспоненциально распределённым, а время от начала до $k$-го успеха — имеющим гамма-распределение $Gamma(k, \lambda)$. На этот раз Вася решил дождаться $k_1$ успеха, затем быстренько пообедать, а затем дождаться ещё $k_2$ успехов. Пусть $X_1$ — время от начала наблюдения до обеда, а $X_2$ — время от обеда до конца опытов. Также введём $S=X_1 + X_2$ и $Z = X_1 / S$ — долю времени до обеда от общего времени набопытов.

\begin{enumerate}
    \item Найдите совместную функцию плотности $S$ и $Z$ с точностью до константы.
    \item Являются ли $S$ и $Z$ независимыми случайными величинами?
    \item Найдите частные функции плотности $S$ и $Z$.
    \item Как называется закон распределения $S$?
        \item Как называется закон распределения $Z$?
    \item Какой закон распределения имеет величина $W = 1 - Z$?
\end{enumerate}

\begin{sol}
\end{sol}
\end{problem}




\begin{problem}
    Вася оценивает регрессию $y$ на регрессоры $X$, включающие константу, 
    а на самом деле все коэффициенты $\beta_j$ кроме константы равны нулю. 
    Ошибки $u_i$ распределены нормально $\cN(0;\sigma^2)$. Какое распределение имеет $R^2$?
\begin{sol}
\end{sol}
\end{problem}






\section{Блок}

\begin{problem}
Найдите матрицу $M$ и укажите размеры всех блоков
\begin{enumerate}
  \item Блок $C$ имеет размер $p\times p$, блок $F$ — размер $q\times q$
    \[M=\begin{pmatrix}
      A & B \\
    \end{pmatrix} \cdot
    \begin{pmatrix}
      C & D \\
      E & F \\
    \end{pmatrix}
  \]
\item Блок $C$ имеет размер $p\times p$, блок $F$ — размер $q\times q$
  \[
    M=\begin{pmatrix}
      C & D \\
      E & F \\
    \end{pmatrix}\cdot
    \begin{pmatrix}
      A \\
      B \\
    \end{pmatrix}
  \]
  \item Блок $C$ имеет размер $p\times p$, блок $B$ — размер $q\times q$
    \[
      M=\begin{pmatrix}
      A & B \\
      C & D \\
    \end{pmatrix}^T
  \]

\end{enumerate}

\begin{sol}
\begin{enumerate}
\item $M = \begin{pmatrix}
AC + BE & AD + BF
\end{pmatrix}
$

Предположим, что матрицы $A$ и $B$ имеют $m$ строк.
Тогда размерность блока $AC + BE$ — $m \times p$, $AD + BF$ — $m \times q$.
\item $M = \begin{pmatrix}
CA + DB \\
EA + FB
\end{pmatrix}
$

Предположим, что матрицы $A$ и $B$ имеют $n$ столбцов.
Тогда размерность блока $CA + DB$ — $p \times n$, $EA + FB$ — $q \times n$.
\item $M = \begin{pmatrix}
A^T & C^T \\
B^T & D^T
\end{pmatrix}
$

Блок $A^T$ имеет размерность $p \times q$, $B^T$ — $q \times q$, $C^T$ — $p \times p$, $D^T$ — $q \times p$.
\end{enumerate}
\end{sol}
\end{problem}

\begin{problem}
Найдите обратную матрицу $M^{-1}$ для каждого из случаев
\begin{enumerate}
  \item Блоки $A_{p\times p}$ и $B_{q\times q}$ обратимы,
    \[
      M = \begin{pmatrix}
	A & 0 \\
	0 & B \\
      \end{pmatrix}
    \]
  \item Блоки $A_{p\times p}$ и $B_{q\times q}$ обратимы,
    \[
      M = \begin{pmatrix}
	0 & A \\
	B & 0 \\
      \end{pmatrix}
    \]

 \item Блоки $A_{p\times p}$ и $B_{q\times q}$ обратимы,
    \[
      M = \begin{pmatrix}
	A & C \\
	0 & B \\
      \end{pmatrix}
    \]
  \item Блоки $A_{p\times p}$ и $B_{q\times q}$ обратимы,
    \[
      M = \begin{pmatrix}
	A & 0 \\
	C & B \\
      \end{pmatrix}
    \]
\end{enumerate}
\begin{sol}
\begin{enumerate}
\item $M^{-1} = \begin{pmatrix}
A^{-1} & 0 \\
0 & B^{-1}
\end{pmatrix}$
\item $M^{-1} = \begin{pmatrix}
0 & B^{-1}  \\
A^{-1} & 0
\end{pmatrix}$
\item $M^{-1} = \begin{pmatrix}
A^{-1} & -A^{-1}CB^{-1}  \\
0 & B^{-1}
\end{pmatrix}$
\item $M^{-1} = \begin{pmatrix}
A^{-1} & 0  \\
-B^{-1}CA^{-1} & B^{-1}
\end{pmatrix}$
\end{enumerate}
\end{sol}
\end{problem}

\begin{problem}
    Блоки $A_{p\times p}$ и $B_{q\times q}$ обратимы, матрица $M$ имеет вид
    \[
      M = \begin{pmatrix}
	A & C \\
	D & B \\
      \end{pmatrix}
    \]
    Рассмотрим обратную матрицу $M^{-1}$
    \[
      M^{-1} = \begin{pmatrix}
	X & Z \\
	Y & W \\
      \end{pmatrix}
    \]
    \begin{enumerate}
      \item Найдите блок $X$ с помощью процедуры Гаусса;
      \item Найдите блок $X$ решив систему двух уравнений на блоки $X$ и $Y$;
      \item Докажите тождество Вудберри
      \[
      (A - CB^{-1}D)^{-1}  =  A^{-1} + A^{-1}C(B - DA^{-1}C)^{-1}DA^{-1}
      \]
    \end{enumerate}
    %\todo[inline]{добавить тождество Вудберри}

\begin{sol}
\begin{enumerate}
\item \begin{multline*}
\left(
\begin{array}{cc|cc}
A & C & I & 0 \\
D & B & 0 & I
\end{array}
\right)
\sim
\left(
\begin{array}{cc|cc}
I & A^{-1}C & A^{-1} & 0 \\
D & B & 0 & I
\end{array}
\right)
\sim
\left(
\begin{array}{cc|cc}
I & A^{-1}C & A^{-1} & 0 \\
0 & B - DA^{-1}C & -DA^{-1} & I
\end{array}
\right)
\sim \\
\left(
\begin{array}{cc|cc}
I & A^{-1}C & A^{-1} & 0 \\
0 & I & -(B - DA^{-1}C)^{-1}DA^{-1} & (B - DA^{-1}C)^{-1}
\end{array}
\right)
\sim \\
\left(
\begin{array}{cc|cc}
I & 0 & A^{-1} + A^{-1}C(B - DA^{-1}C)^{-1}DA^{-1}   & -A^{-1}C(B - DA^{-1}C)^{-1} \\
0 & I & -(B - DA^{-1}C)^{-1}DA^{-1} & (B - DA^{-1}C)^{-1}
\end{array}
\right)
\end{multline*}
То есть $X = A^{-1} + A^{-1}C(B - DA^{-1}C)^{-1}DA^{-1}$.
\item Из равенства
\[
\begin{pmatrix}
A & C \\
D & B
\end{pmatrix}
\begin{pmatrix}
X & Z \\
Y & W
\end{pmatrix}
=
\begin{pmatrix}
I & 0 \\
0 & I
\end{pmatrix}
\]
получаем систему:
\begin{align*}
\begin{cases}
AX + CY = I \\
DX + BY = 0
\end{cases}
\Rightarrow
\begin{cases}
X  = A^{-1}(I-CY) \\
DX + BY = 0
\end{cases}
\end{align*}
Подставляя первое уравнение во второе, получим:
\[
D A^{-1}(I-CY) = -BY \Rightarrow DA^{-1} = (DA^{-1}C- B)Y \Rightarrow I = (C - AD^{-1}B)Y \Rightarrow Y = (C - AD^{-1}B)^{-1}
\]
И окончательно из второго уравнения:
\[
DX = -B (C - AD^{-1}B)^{-1} \Rightarrow -(C - AD^{-1}B)B^{-1}DX = I \Rightarrow X = (A - CB^{-1}D)^{-1}
\]
\item
\begin{multline*}
(A - CB^{-1}D)(A^{-1} + A^{-1}C(B-DA^{-1}C)^{-1}DA^{-1}) = \\
I - CB^{-1}DA^{-1} + (C - CB^{-1}DA^{-1}C)(B-DA^{-1}C)^{-1}DA^{-1} = \\
I - CB^{-1}DA^{-1} + CB^{-1}(B - DA^{-1}C)(B-DA^{-1}C)^{-1}DA^{-1} = \\
I - CB^{-1}DA^{-1} + CB^{-1}DA^{-1} = I
\end{multline*}
\end{enumerate}
\end{sol}
\end{problem}








\section{Максимально правдоподобно}

\begin{problem}
Рассмотрим модель регрессии с одним параметром, $y_i = \beta x_i + u_i$,
где $x_i$ неслучайны и не все равны нулю, а $u_i$ нормальны $\cN(0;\sigma^2)$ и независимы.

Рассмотрим варианты предпосылок:

\begin{enumerate}
    \item Величина $\sigma^2$ известна,
    и исследователь хочет проверить гипотезу $H_0$: $\beta = 7$ против $H_a$: $\beta \neq 7$.
    \item Величина $\beta$ известна,
    и исследователь хочет проверить гипотезу $H_0$: $\sigma^2 = 1$ против $H_a$: $\sigma^2 \neq 1$.
    \item Величины $\sigma^2$ и $\beta$ неизвестны,
    и исследователь хочет проверить гипотезу $H_0$: $\beta = 7$ против $H_a$: $\beta \neq 7$.
    \item Величины $\sigma^2$ и $\beta$ неизвестны,
    и исследователь хочет проверить гипотезу $H_0$: $\beta = 7$, $\sigma^2 = 1$ против $H_a$: $\beta \neq 7$ или $\sigma^2 \neq 1$.
\end{enumerate}

Для каждого варианта предпосылок:

\begin{enumerate}
    \item Найдите функцию правдоподобия $\ell(\theta)$, её градиент $s(\theta)$,
    матрицу Гессе $H(\theta)$, теоретическую информацию Фишера $I(\theta)$.
    \item Найдите $\hat\theta_{UR}$, $\hat\theta_R$, $\ell(\hat\theta_{UR})$,
    $\ell(\hat\theta_R)$, $s(\hat\theta_{UR})$, $s(\hat\theta_R)$.
    \item Выпишите формулу для $RSS_R$ и $RSS_{UR}$.
    \item Найдите оценку информации Фишера $\hat I_R$, $\hat I_{UR}$
    подставив в теоретическую информацию Фишера оценённые параметры.
    \item Выведите формулы для $LR$, $LM$ и $W$ статистики.
    Можно выражать их через $RSS_R$ и $RSS_{UR}$.
    \item Упорядочьте статистики по возрастанию.
\end{enumerate}


\begin{sol}

\end{sol}
\end{problem}




\begin{problem}
  Величины $y_1$, $y_2$, \ldots, $y_n$ независимы и экспоненциально распределены с параметром $\lambda$. По выборке из $100$ наблюдений оказалось, что $\sum y_i=200$. Исследователь Андреас хочет проверить гипотезу $H_0$: $\E(y_i)=1$ против альтернативной $\E(y_i)\neq 1$.

\begin{enumerate}
  \item Выпишите логарифмическую функцию правдоподобия $\ell(\lambda)$;
  \item Найдите оценку $\hat \lambda$ методом максимального правдоподобия
    в общем виде и для имеющейся выборки;
  \item Найдите теоретическую информацию Фишера $I(\lambda)$ для $n$ наблюдений;
  \item Выведите формулы для статистик отношения правдоподобия, множителей Лагранжа и Вальда в общем виде;
  \item Найдите значения статистик отношения правдоподобия, множителей Лагранжа и Вальда для имеющейся выборки;
  \item Проверьте гипотезу $H_0$ с помощью трёх статистик.
\end{enumerate}
\begin{sol}
\begin{enumerate}
\item $\ell = n \ln \lambda - \lambda \sum y_i$
\item $\hat \lambda = \frac{\sum y_i}{n} = \frac{1}{2}$
\item $I(\lambda)=\frac{n}{\lambda^2}$
\item $LR = 2(n \ln \frac{n}{\sum y_i} - n - n \ln \lambda_R + \lambda_R \sum y_i)$

$LM = \left(\frac{n}{\lambda} - \sum y_i \right)^2 \frac{\lambda^2}{n}$

$W = \left(\frac{\sum y_i}{n} - \lambda_R \right)^2 \frac{n}{\lambda^2}$
\item $LR \approx 61.37$, $LM = W = 100$
\item $\chi^2_{1, 0.95} = 3.84$, основная гипотеза отвергается.
\end{enumerate}
\end{sol}
\end{problem}



\begin{problem}
  Рассмотрим модель простой регрессии $y_i = \beta x_i + u_i$,
  где ошибки $u_i$ независимы и имеют стандартное нормальное распределение, $u_i \sim \cN(0;1)$.
  По выборке из 100 наблюдений оказалось,
  что $\sum x_i^2 = 100$, $\sum y_i^2=900$,  а $\sum y_i x_i  = 250$.
  Исследователь Рамирес хочет проверить $H_0$: $\beta=0$.


\begin{enumerate}
  \item Выпишите логарифмическую функцию правдоподобия $\ell(\beta)$;
  \item Найдите оценку $\hat \beta$ методом максимального правдоподобия
    в общем виде и для имеющейся выборки;
  \item Найдите теоретическую информацию Фишера $I(\beta)$ для $n$ наблюдений;
  \item Выведите формулы для статистик отношения правдоподобия, множителей Лагранжа и Вальда в общем виде;
  \item Найдите значения статистик отношения правдоподобия, множителей Лагранжа и Вальда для имеющейся выборки;
  \item Проверьте гипотезу $H_0$ с помощью трёх статистик.
\end{enumerate}

\begin{sol}
\begin{enumerate}
\item $\ell = n \ln \sqrt{2\pi} - \frac{1}{2} \sum (y_i - \beta x_i)^2$
\item $\hb_{ML} = \frac{\sum y_i x_i}{\sum x_i^2} = 2.5$
\item $I(\beta) = \sum x_i^2$
\item $LR = -\sum (y_i - \hb_{ML} x_i)^2 + \sum (y_i - \beta_{R} x_i)^2$

$LM = (\sum (y_i x_i - \beta_R x_i^2))^2 \cdot \frac{1}{\sum x_i^2}$

$W = (\hb_{ML} - \beta_R)^2 \sum x_i^2$
\item $LR = LM = W = 625$
\item $\chi^2_{1, 0.95} = 3.84$, основная гипотеза отвергается.
\end{enumerate}
\end{sol}
\end{problem}



\begin{problem}
Исследовательница Геральдина заглядывает $n$ раз в случайные аудитории бывшей шпульно-катушечной фабрики. В каждой аудитории независимо от других идёт семинар по теории вероятностей, эконометрике, микро или макро. Пусть $p_1$, $p_2$, $p_3$ — это вероятности семинаров по теории вероятностей, эконометрике и микро. Вероятность семинара по макро мы отдельным параметром не вводим, так как иначе параметры будут зависимы и нужно будет искать ограниченный экстремум правдоподобия. Пусть $y_1$, $y_2$, $y_3$ — количество попаданий Геральдины на теорию вероятностей, эконометрику и микро.

По выборки из $100$ наблюдений оказалось, что $y_1=20$, $y_2=30$, $y_3=20$. Геральдина предполагает, что все четыре дисциплины равновероятны.

\begin{enumerate}
  \item Выпишите логарифмическую функцию правдоподобия $\ell(\p)$;
  \item Найдите оценку $\hat p$ методом максимального правдоподобия
    в общем виде и для имеющейся выборки;
  \item Найдите теоретическую информацию Фишера $I(p)$ для $n$ наблюдений;
  \item Найдите явно $I^{-1}(p)$;
  \item Выведите формулы для статистик отношения правдоподобия, множителей Лагранжа и Вальда в общем виде;
  \item Найдите значения статистик отношения правдоподобия, множителей Лагранжа и Вальда для имеющейся выборки;
  \item Проверьте гипотезу $H_0$ с помощью трёх статистик на уровне значимости 5\%.
  \item (*) Обобщиет формулы трёх статистик на случай произвольного количества дисциплин и произвольной гипотезы $H_0$: $p=p^0$.
\end{enumerate}


\begin{sol}
\begin{enumerate}
\item $\ell = const + y_1 \ln p_1 + y_2 \ln p_2 + y_3 \ln p_3 + (n - y_1 - y_2 - y_3) \ln(1 - y_1 - y_2 - y_3)$
\item $\hat p = \begin{pmatrix} y_1 / n \\ y_2 / n \\ y_3 / n \end{pmatrix} = \begin{pmatrix} 0.2 \\ 0.3 \\ 0.2 \end{pmatrix}$
\item $I(p) = \begin{pmatrix}
\frac{n}{p_1} + \frac{n}{1-p_1-p_2-p_3} & \frac{n}{1-p_1-p_2-p_3} & \frac{n}{1-p_1-p_2-p_3} \\
\frac{n}{1-p_1-p_2-p_3} & \frac{n}{p_2} + \frac{n}{1-p_1-p_2-p_3} & \frac{n}{1-p_1-p_2-p_3} \\
\frac{n}{1-p_1-p_2-p_3} & \frac{n}{1-p_1-p_2-p_3} & \frac{n}{p_3} + \frac{n}{1-p_1-p_2-p_3}
\end{pmatrix}$
\item $I^{-1}(p) = \begin{pmatrix}
\frac{p_1(1-p_1)}{n} & -\frac{p_1 p_2}{n} & -\frac{p_1 p_3}{n} \\
-\frac{p_1 p_2}{n} & \frac{p_2(1-p_2)}{n} & -\frac{p_2 p_3}{n} \\
-\frac{p_1 p_3}{n} & -\frac{p_2 p_3}{n} & \frac{p_3(1-p_3)}{n}
\end{pmatrix}$
\end{enumerate}
\end{sol}
\end{problem}


\begin{problem}
  Логарифмическая функция правдоподобия имеет вид
  \[
     \ell(\theta) = a - \frac{1}{2}(\theta - h(y))' Q (\theta - h(y)),
  \]
  где $Q$ — постоянная симметричная матрица, а $h(y)$ — функция от выборки.
  Вектор параметров $\theta$ состоит из двух блоков, а матрица $Q$ — из четырёх блоков
  \[
    \theta = \begin{pmatrix}
      \theta_1 \\
      \theta_2 \\
    \end{pmatrix}, \quad
    Q = \begin{pmatrix}
      A & B \\
      B^T & C \\
    \end{pmatrix}
  \]
Настырный исследователь Никанор хочет проверить гипотезу $H_0$: $\theta_1 = \theta_1^0$ про часть параметров, входящих в вектор $\theta$;


  \begin{enumerate}
    \item Найдите неограниченную оценку метода максимального правдоподобия $\hat \theta^{UR}$;
    \item Найдите ограниченную оценку метода максимального правдоподобия $\hat \theta^{R}$;
    \item Выведите формулу для $LR$ статистики;
    \item Выведите формулу для $LM$ статистики;
    \item Выведите формулу для $W$ статистики;
    \item Какие из указанных формул равны?
  \end{enumerate}


\begin{sol}
\begin{enumerate}
\item $\hat \theta^{UR} = h(y)$
\item $\theta^{R} = \begin{pmatrix}
\theta_1^0 \\
h_2(y) - C^{-1}B^T(\theta_1^0 - h_1(y))
\end{pmatrix}$
\item[3—6.] $LR=LM=W= (\theta_1^0 - h_1(y))^T(A-BC^{-1}B^T)(\theta_1^0 - h_1(y))$
\end{enumerate}
\end{sol}
\end{problem}


\begin{problem}
  Рассмотрим модель множественной регрессии $y = X\beta + u$, где регрессоры детерминистические,
  ошибки имеют многомерное нормальное распределение,
  а ковариационная матрица $\Var(u)$ единичная. Разобьём вектор коэффициентов $\beta$ на две части
  \[
    \beta = \begin{pmatrix}
      \beta_1 \\
      \beta_2 \\
    \end{pmatrix}
  \]

  \begin{enumerate}
    \item Докажите, что логарифмическая функция правдоподобия представима в виде
\[
     \ell(\theta) = a - \frac{1}{2}(\theta - h(y))' Q (\theta - h(y)),
  \]
\item Явно найдите матрицу $Q$ и функцию $h(y)$;
\item Выведите формулу для $LR$, $LM$ и теста Вальда для проверки гипотезы $H_0$: $\beta_1 = \beta_1^0$;
\item Как найденная формула отличается от обычной $F$ статистики?
\item Как найденная формула упрощается для случая проверки гипотезы о незначимости регрессии в целом?
  \end{enumerate}

\begin{sol}
\begin{enumerate}
\item[2.] $Q = \frac{1}{\sigma^2} X^T X$, $h(y) = (X^T X)^{-1} X^T y$
\end{enumerate}
\end{sol}
\end{problem}



\begin{problem}
  Рассмотрим модель множественной регрессии $y = X\beta + u$, где регрессоры детерминистические,
  ошибки имеют многомерное нормальное распределение,
  а ковариационная матрица $\Var(u)=\sigma^2 I$. Разобьём вектор коэффициентов $\beta$ на две части
  \[
    \beta = \begin{pmatrix}
      \beta_1 \\
      \beta_2 \\
    \end{pmatrix}
  \]

  \begin{enumerate}
    \item Докажите, что логарифмическая функция правдоподобия представима в виде
\[
     \ell(\theta) = a + (\theta - h(y))' Q (\theta - h(y)),
  \]
\item Явно найдите матрицу $Q$ и функцию $h(y)$;
\item Выведите формулы для $LR$, $LM$ и теста Вальда для проверки гипотезы $H_0$: $\beta_1 = \beta_1^0$;
\item Как найденные формулы отличается от обычной $F$ статистики?
\item Как найденные формулы упрощается для случая проверки гипотезы о незначимости регрессии в целом?
  \end{enumerate}

  \begin{sol}
  \end{sol}
\end{problem}


\begin{problem}
Рассмотрим $LR$, $LM$ и $W$ статистики в задаче оценки параметров модели $y=X\beta + u$, с нормальными ошибками $u\sim \cN(0;\sigma^2 \cdot I)$ и неизвестной $\sigma^2$.

Известно, что при проверке гипотезы о линейных ограничениях на $\beta$ оказывается, что
\[
\begin{cases}
   LR = n \ln s \\
   W = n (s - 1) \\
   LM = n(s-1)/s \\
\end{cases},
\]
где $s = RSS_R / RSS_{UR}$.

Докажите, что $LM \leq LR \leq W$.
  \begin{sol}
  \end{sol}
\end{problem}


\begin{problem}
Рассмотрим $LR$, $LM$ и $W$ статистики в задаче оценки параметров модели $y=X\beta + u$, с нормальными ошибками $u\sim \cN(0;\sigma^2 \cdot I)$ и неизвестной $\sigma^2$.

Известно, что при проверке гипотезы о линейных ограничениях на $\beta$ оказывается, что
\[
\begin{cases}
   LR = n \ln s \\
   W = n (s - 1) \\
   LM = n(s-1)/s \\
\end{cases},
\]
где $s = RSS_R / RSS_{UR}$.

Эконометрэсса Фиалка по 60 наблюдениям проверяет гипотезу о равенстве пяти параметров нулю
в регрессии с десятью параметрами $\beta$ при 5\%-м уровне значимости.

Найдите точные критические значения для $LR$, $LM$ и $W$ и сравните их с асимптотическими.

  \begin{sol}
      Мы знаем, что $(RSS_R - RSS_{UR}) \cdot (60 - 10) / 5 RSS_{UR}$ имеет в точности $F$-распределение.
      Находим критическое значение для него по таблице. Выражаем три статистики через $F$-распределение.
      Получаем точные критические значения.
  \end{sol}
\end{problem}







\section{Гетероскедастичность}


\begin{problem}
  Имеeтся три наблюдения

  \begin{tabular}{cccc}
    \toprule
  $x_i$ & $1$ & $2$ & $2$ \\
  $y_i$ & $1$ & $2$ & $3$ \\
\bottomrule
  \end{tabular}

  Экономэтр Антоний хочет оценить зависимость $y_i = \beta x_i + u_i$.

  \begin{enumerate}
    \item Найдите оценку $\hb$ с помощью МНК;
    \item Найдите стандартную ошибку $se(\hb)$ предполагая гомоскедастичность;
    \item Найдите робастные к гетероскедастичности стандартные ошибки $se_{HC0}(\hb)$ и $se_{HC3}(\hb)$;
    \item Найдите эффективную оценку $\hb$, если дополнительно известно, что $\Var(u_i|x_i)=\sigma^2(3x_i-2)$;
    \item Найдите эффективную оценку $\hb$, если дополнительно известно, что
      \[
    \Var(u|X) = \begin{pmatrix}
      4 \sigma^2 & -\sigma^2 & 0 \\
      -\sigma^2 & 9\sigma^2 & 0 \\
      0 & 0 & \sigma^2 \\
    \end{pmatrix}
  \]
  \end{enumerate}

\begin{sol}
\begin{enumerate}
\item $\hb_{OLS} = 11/9$
\item $se(\hb) = \sqrt{5/162}$
\item $se_{HC0}(\hb) = \sqrt{168}/81$, $se_{HC3}(\hb) = \sqrt{2649}/180$
\item $\hb = 7/6$
\end{enumerate}
\end{sol}
\end{problem}


\begin{problem}
Известно, что после деления каждого уравнения регрессии $y_i = \beta_1 + \beta_2 x_i + \e_i$ на $x_i^2$ гетероскедастичность ошибок была устранена. Какой вид имела дисперсия ошибок, $\Var(\e_i)$?

\begin{sol}
$\Var(\e_i)=cx_i^4$
\end{sol}
\end{problem}



\begin{problem}
Для линейной регрессии $y_i = \beta_1 + \beta_2 x_i + \beta_3 z_i + \e_i$ была
выполнена сортировка наблюдений по возрастанию переменной $x$. Исходная модель оценивалась по разным частям выборки:

\begin{tabular}{c|cccc}
\toprule
Выборка & $\hb_1$ & $\hb_2$ & $\hb_3$ & $RSS$ \\
\midrule
$i=1,\ldots, 30$ & $1.21$ & $1.89$ & $2.74$ & $48.69$ \\
$i=1,\ldots, 11$ & $1.39$ & $2.27$ & $2.36$ & $10.28$ \\
$i=12,\ldots, 19$ & $0.75$ & $2.23$ & $3.19$ & $5.31$ \\
$i=20,\ldots, 30$ & $1.56$ & $1.06$ & $2.29$ & $14.51$ \\
\bottomrule
\end{tabular}

Известно, что ошибки в модели являются независимыми нормальными случайными величинами с нулевым математическим ожиданием. Протестируйте
ошибки на гетероскедастичность на уровне значимости 5\%.



\begin{sol}
Протестируем гетероскедастичность ошибок при помощи теста Голдфельда-
Квандта. $H_0: \Var(\e_i)=\sigma^2$, $H_a: \Var(\e_i)=f(x_i)$

\begin{enumerate}
\item Тестовая статистика $GQ=\frac{RSS_3/(n_3-k)}{RSS_1/(n_1-k)}$, где $n_1=11$ — число наблюдений в первой подгруппе, $n_3=11$ — число наблюдений в
последней подгруппе, $k=3$ — число факторов в модели, считая единичный столбец.
\item Распределение тестовой статистики при верной $H_0$: $GQ\sim F_{n_3-k,n_1-k}$
\item Наблюдаемое значение $GQ_{obs}=1.41$
\item Область, в которой $H_0$ не отвергается: $GQ\in [0;3.44]$
\item Статистический вывод: поскольку $GQ_{obs} \in [0;3.44]$, то на основании имеющихся наблюдений на уровне значимости 5\% основная гипотеза $H_0$ не может быть отвергнута. Таким образом, тест Голдфельда-Квандта не выявил гетероскедастичность.
\end{enumerate}
\end{sol}
\end{problem}



\begin{problem}
Рассмотрим линейную регрессию $y_i = \beta_1 + \beta_2 x_i + \beta_3 z_i + \e_i$ по 50 наблюдениям. При оценивании с помощью МНК были получены результаты: $\hb_1=1.21$, $\hb_2=1.11$, $\hb_3=3.15$, $R^2=0.72$.

Оценена также вспомогательная регрессия: $\he^2_i=\delta_1+\delta_2 x_i +\delta_3 z_i+\delta_4 x_i^2+\delta_5 z_i^2+\delta_6 x_i z_i + u_i$. Результаты оценивания следующие: $\hat{\delta}_1=1.50$, $\hat{\delta}_2=-2.18$,  $\hat{\delta}_3=0.23$,  $\hat{\delta}_4=1.87$,  $\hat{\delta}_5=-0.56$,  $\hat{\delta}_6=-0.09$,  $R^2_{aux}=0.36$


Известно, что ошибки в модели являются независимыми нормальными случайными величинами с нулевым математическим ожиданием. Протестируйте
ошибки на гетероскедастичность на уровне значимости 5\%.


\begin{sol}
Протестируем гетероскедастичность ошибок при помощи теста Уайта. $H_0: \Var(\e_i)=\sigma^2$, $H_a: \Var(\e_i)=\delta_1+\delta_2 x_i +\delta_3 z_i+\delta_4 x_i^2+\delta_5 z_i^2+\delta_6 x_i z_i$.
\begin{enumerate}
\item Тестовая статистика $W=n\cdot R^2_{aux}$, где $n$ — число наблюдений, $R^2_{aux}$ — коэффициент детерминации для вспомогательной регрессии.
\item Распределение тестовой статистики при верной $H_0$: $W\sim \chi^2_{k_{aux}-1}$, где $k_{aux}=6$ — число регрессоров во вспомогательной регрессии, считая константу.
\item Наблюдаемое значение тестовой статистики: $W_{obs}=18$
\item Область, в которой $H_0$ не отвергается: $W\in [0;W_{crit}]=[0;11.07]$
\item Статистический вывод: поскольку $W_{obs} \notin [0;11.07]$, то на основании имеющихся наблюдений на уровне значимости 5\% основная гипотеза $H_0$ отвергается. Таким образом, тест Уайта выявил гетероскедастичность.
\end{enumerate}
\end{sol}
\end{problem}

\begin{problem}
Найдите число коэффициентов во вспомогательной регрессии, необходимой для выполнения теста Уайта, если число коэффициентов в исходной регрессии равно $k$, включая свободный член.

\begin{sol}
$k(k+1)/2$
\end{sol}
\end{problem}

\begin{problem}
Рассмотрим модель регрессии $y_i=\beta_1+\beta_2 x_i + \beta_3 z_i+\e_i$, в которой
ошибки $\e_i$ независимы и имеют нормальное распределение $N(0,\sigma^2)$. Для $n = 200$ наблюдений найдите
\begin{enumerate}
\item вероятность того, что статистика Уайта окажется больше 10;
\item ожидаемое значение статистики Уайта;
\item дисперсию статистики Уайта.
\end{enumerate}


\begin{sol}
$0.0752$, $5$, $10$
\end{sol}
\end{problem}


\begin{problem}
Рассматривается модель $y_t=\beta_1+\e_t$, где ошибки $\e_t$  — независимые
случайные величины с $\E(\e_t)=0$ и $\Var(\e_t)=t$. Найдите наиболее эффективную
оценку неизвестного параметра $\beta_1$ в классе линейных по $y$ и несмещённых оценок.

\begin{sol}
\end{sol}
\end{problem}

\begin{problem}
  Экономэтр Антоний исследует зависимость надоя коров в литрах в год, $y_i$, от дамми-переменной $x_i$, отвечающей за прослушивание коровами ежедневно Девятой симфонии, $y_i = \beta_1 + \beta_2 x_i + u_i$. Антоний раздобыл следующие данные:

  \begin{tabular}{cccc}
    \toprule
    Подвыборка & Размер & $\sum y_i$ & $\sum y_i^2$ \\
    \midrule
    $x_i = 0$ & $n_0 = 100$ & 200 & 4000 \\
    $x_i = 1$ & $n_1 = 100$ & 300 & 5000 \\
\bottomrule
  \end{tabular}

  \begin{enumerate}
 \item Найдите МНК-оценки $\beta_1$ и $\beta_2$;
 \item Постройте 95\%-ый доверительный интервал для $\beta_2$ предполагая гомоскедастичность $u_i$;
 \item Найдите робастную к гетероскедастичности оценку $\hVar_{HC0}(\hb)$;
 \item Найдите робастную к гетероскедастичности оценку $\hVar_{HC3}(\hb)$;
 \item Постройте 95\%-ый доверительный интервал для $\beta_2$ с помощью скорректированной $se_{HC0}(\hb_2)$;
 \item Дополнительно предположив, что $\Var(u_i |x_i) = \sigma^2(1+3x_i)$, найдите эффективную оценку $\hb_2$ и постройте доверительный интервал для неё.
   \end{enumerate}
  \begin{sol}

  \end{sol}
\end{problem}



\begin{problem}
Эконометресса Прасковья использует традиционную оценку ковариационной матрицы, а эконометресса Мелони — оценку Уайта.

Какие оценки дисперсии $\hb_1$ и формулы для $t$-статистики получат Прасковья и Мелони в модели $y_i = \beta_1 + u_i$?
\begin{sol}
Одинаковые.
\end{sol}
\end{problem}

\begin{problem}
Эконометресса Прасковья использует традиционную оценку ковариационной матрицы, а эконометресса Мелони — оценку Уайта.

Обе эконометрессы оценивают модель $y_i = \beta_1 + \beta_2 d_i + u_i$, где $d_i$ — дамми-переменная, равна 0 или 1. Дамми-переменная делит выборку на две части. Обозначим количество наблюдений в «нулевой» части как $n_0$, среднее — как $\bar y_0$, и общую сумму квадратов — как $TSS_0$. Аналогичные величины для «единичной» части выборки — $n_1$, $\bar y_1$ и $TSS_1$. И для всей выборки — $n$, $\bar y$, $TSS$.


  \begin{enumerate}
    \item Найдите оценки $\hb_1$ и $\hb_2$.
    \item Найдите оценки $\hVar(\hb_1)$ и $\hVar_W(\hb_1)$. Верно ли, что $\hVar_W(\hb_1) \geq \hVar(\hb_1)$?
    \item Найдите оценки $\hVar(\hb_2)$ и $\hVar_W(\hb_2)$. Верно ли, что $\hVar_W(\hb_2) \geq \hVar(\hb_2)$?
    \item Найдите оценки $\hCov(\hb_1, \hb_2)$ и $\hCov_W(\hb_1, \hb_2)$.
    \item Найдите оценки $\hVar(\hb_1 + \hb_2)$ и $\hVar_W(\hb_1 + \hb_2)$.
  \end{enumerate}

\begin{sol}
\[
\hVar(\hb_1) = TSS \frac{1}{n_0}\frac{1}{n-2}
\]
\[
\hVar(\hb_1) = TSS_0\frac{n}{n_0} \frac{1}{n_0}\frac{1}{n-2}
\]
\end{sol}
\end{problem}



\begin{problem}
В модели $y_i=\beta x_i+\e_i$ предполагается гетероскедастичность вида $\Var(\e_i)=\exp(\gamma_1+\gamma_2 x_i)$ и нормальность ошибок.
\begin{enumerate}
\item Сформулируйте гипотезу о гомоскедастичности с помощью коэффициентов.
\item Выведите в явном виде оценку максимального правдоподобия при предположении о гомоскедастичности.
\item Выпишите условия первого порядка для оценки максимального правдоподобия без предположения о гомоскедастичности.
\item Выведите в явном виде формулу для LM теста множителей Лагранжа.
\end{enumerate}


\begin{sol}
В предположении о гомоскедастичности, $\gamma_2=0$, оценка правдоподобия совпадает с МНК-оценкой, значит $\hb=\sum y_i x_i/ \sum x_i^2$. И $\hs^2_i=RSS/n$, значит $\hat{\gamma_1}=\ln(RSS/n)$.
\end{sol}
\end{problem}

\begin{problem}
Для регрессии $y = X\beta + \e$ с $\E(\e) = 0$, $\Var(\e) = \Sigma \neq \sigma^2 I$, оцененной с помощью обобщённого метода наименьших квадратов, найдите ковариационную матрицу $\Cov(\hb_{GLS}, \e)$



\begin{sol}
\begin{multline*}
\Cov(\hb_{GLS}, \e) = \Cov \left( (X' \Sigma^{-1} X)^{-1} X' \Sigma^{-1} y, \e \right) = \\
= \Cov \left( (X' \Sigma^{-1} X)^{-1} X' \Sigma^{-1} \e, \e \right) = \\
= (X' \Sigma^{-1} X)^{-1} X' \Sigma^{-1} \Cov(\e, \e) =\\
= (X' \Sigma^{-1} X)^{-1} X' \Sigma^{-1} \Sigma = (X' \Sigma^{-1} X)^{-1} X'
\end{multline*}
\end{sol}
\end{problem}


\begin{problem}
В оригинальном тесте Бройша-Пагана на гетероскедастичность два шага. Сначала строится основная регрессия $y_i$ на некоторые регрессоры и получаются остатки $\hat u_i$. На втором шаге строится регрессия квадрата остатков $\hat u_i$ на переменные, от которых потенциально зависит условная дисперсия $\Var(u_i|Z)$. Статистика Бройша-Пагана считается как $BP=ESS/2$, где $ESS$ — объяснённая сумма квадратов регрессии второго шага. Оригинальный тест Уайта считается как $W=nR^2$, где $R^2$ — коэффициент детерминации регрессии второго шага.
\begin{enumerate}
  \item Найдите отношение $\frac{nR^2}{ESS/2}$;
  \item Найдите предел по вероятности $\plim \frac{nR^2}{ESS/2}$;
  \item Какое распределение имеют статистики $BP$ и $W$?
  \item Какой вид имеет статистика множителей Лагранжа?
\end{enumerate}

\todo[inline]{распотрошить статью BP на задачу, статья о похожести BP и W, отдельно Коэнкера про студентизированную версию}

\begin{sol}

  \end{sol}
\end{problem}





\section{Логит, пробит и хоббит!}


\begin{problem}
Бандерлог из Лога оценил логистическую регрессию по четырём наблюдениям и одному признаку с константой, получил $b_i = \hat\P(y_i = 1|x_i)$, но потерял последнее наблюдение:

\begin{tabular}{cc}
  \toprule
  $y_i$ & $b_i$ \\
  \midrule
  1 & 0.7 \\
  -1 & 0.2 \\
  -1 & 0.3 \\
  ? &  ? \\
  \bottomrule
\end{tabular}

\begin{enumerate}
\item Выпишите функцию правдоподобия для задачи логистической регрессии.
\item Выпишите условие первого порядка по коэффициенту перед константой.
\item Помогите бандерлогу восстановить пропущенные значения!
\end{enumerate}

\begin{sol}
$\hat\P(y_i = 1|x_i) = \frac{1}{1+\exp(-\beta_1 - \beta_2 x_i)}$
\begin{enumerate}
\item $loss(\beta_1, \beta_2) = - \sum_{i=1}^l \left([y_i = 1] \ln  \frac{1}{1+\exp(-\beta_1 - \beta_2 x_i)} + [y_i = -1] \ln \left(1 - \frac{1}{1+\exp(-\beta_1 - \beta_2 x_i)}\right)\right)$
\item $\frac{\partial loss}{\partial \beta_1} = - \sum_{i=1}^l \left([y_i = 1] \cdot \frac{1}{1 + \exp(\beta_1 + \beta_2 x_i)} + [y_i = -1] \cdot (-1) \cdot \frac{1}{1 + \exp(-\beta_1 - \beta_2 x_i)} \right)$
\item $y_4 = 1$, $x_4 = 0.8$
\end{enumerate}
\end{sol}
\end{problem}


\begin{problem}
Рассмотрим логистическую функцию $\Lambda(w) = e^w / (1 + e^w)$.
\begin{enumerate}
\item Как связаны между собой $\Lambda(w)$ и  $\Lambda(-w)$?
\item Как связаны между собой $\Lambda'(w)$ и  $\Lambda'(-w)$?
\item Постройте графики функций $\Lambda(w)$ и $\Lambda'(w)$.
\item Найдите $\Lambda(0)$, $\Lambda'(0)$, $\ln\Lambda(0)$.
\item Найдите обратную функцию $\Lambda^{-1}(p)$.
\item Как связаны между собой $\frac{d\ln\Lambda(w)}{dw}$ и $\Lambda(-w)$?
\item Как связаны между собой $\frac{d\ln\Lambda(-w)}{dw}$ и $\Lambda(w)$?
\item Разложите $h(\beta_1, \beta_2)=\ln\Lambda(y_i(\beta_1 + \beta_2 x_i))$ в ряд Тейлора до второго порядка в окрестности точки $\beta_1=0$, $\beta_2=0$.
\end{enumerate}


\begin{sol}
\begin{enumerate}
\item $\Lambda(w) + \Lambda(-w) = 1$
\item $\Lambda'(w) = -\Lambda'(-w)$
\item
\item $\Lambda(0) = 0.5$, $\Lambda'(0) = 0.25$, $\ln\Lambda(0) = -\ln 2$
\item $\Lambda^{-1}(p) = \ln \frac{p}{1-p}$
\item $\frac{d\ln\Lambda(w)}{dw} = \Lambda(-w)$
\item $\frac{d\ln\Lambda(-w)}{dw} = -\Lambda(w)$
\item
\end{enumerate}
\end{sol}
\end{problem}






\begin{problem}
Винни-Пух знает, что мёд бывает правильный, $honey_i=1$, и неправильный, $honey_i=0$. Пчёлы также бывают правильные, $bee_i=1$, и неправильные, $bee_i=0$. По 100 своим попыткам добыть мёд Винни-Пух составил таблицу сопряженности:

\begin{tabular}{c|cc}
\toprule
 & $honey_i=1$ & $honey_i=0$ \\
\midrule
$bee_i=1$ & 12 & 36 \\
$bee_i=0$ & 32 & 20 \\
\bottomrule
\end{tabular}

Винни-Пух использует логистическую регрессию с константой для прогнозирования правильности мёда с помощью правильности пчёл.

\begin{enumerate}
\item Какие оценки коэффициентов получит Винни-Пух?
\item Какой прогноз вероятности правильности мёда при встрече с неправильными пчёлами даёт логистическая модель? Как это число можно посчитать без рассчитывания коэффициентов?
\item Проверьте гипотезу о том, что правильность пчёл не оказывает влияние на правильность мёда с помощью тестов LR, LM и W.
\end{enumerate}

\begin{sol}
\begin{enumerate}
\item Выпишем аппроксимацию функции потерь:
\[
loss(\beta_1, \beta_2) \approx 100 \ln 2 + 6 \beta_1 + 12 \beta_2 + \frac{1}{2}(25 \beta_1^2 + 2 \cdot 12 \beta_1 \beta_2 + 12 \beta_2^2) \to \min_{\beta_1, \beta_2}
\]
Взяв производные по $\beta_1$ и $\beta_2$, получим $\hb_1 = \frac{6}{13}$, $\hb_2 = - \frac{19}{13}$.
\item $\hat{P}(honey_i = 1 | bee_i = 0) = \frac{1}{1+\exp(-6/13)} \approx 0.615$.

Это же число можно было получить из таблицы: $\frac{32}{32 + 20} \approx 0.61$.
\end{enumerate}
\end{sol}
\end{problem}


\begin{problem}
Винни-Пух оценил логистическую регрессию для прогнозирования правильности мёда от высоты дерева (м) $x_i$ и удалённости от дома (км) $z_i$: $\ln odds_i = 2+0.3x_i - 0.5z_i$.
\begin{enumerate}
\item Оцените вероятность того, что $y_i=1$ для $x=15$, $z=3.5$.
\item Оцените предельный эффект увеличения $x$ на единицу на вероятность того, что $y_i=1$ для $x=15$, $z=3.5$.
\item При каком значении $x$ предельный эффект увеличения $x$ на единицу в точке $z=3.5$ будет максимальным?
\end{enumerate}

\begin{sol}
Предельный эффект максимален при максимальной производной $\Lambda'(\hat \beta_1 + \hat\beta_2x + \hat\beta_3z)$, то есть при $\hat \beta_1 + \hat\beta_2x + \hat\beta_3z=0$.
\end{sol}
\end{problem}


\begin{problem}
  Придумайте такие три наблюдения для парной логистической регрессии, чтобы все $x_i$ были разными, не все $y_i$ были одинаковые, а оценки логит-модели не существовали.

  Какое решение задачи этой проблемы разумно предложить при большом количестве наблюдений?

\begin{sol}
  Ввести штраф в жанре LASSO или гребневой регрессии.
\end{sol}
\end{problem}


\begin{problem}
При оценке логит модели
\(
\P(y_i=1)=\Lambda(\b_1+\b_2 x_i)
\)
по 500 наблюдениям оказалось, что $\hb_1=0.7$ и $\hb_2=3$. Оценка ковариационной матрицы коэффициентов имеет вид
\[
\begin{pmatrix}
  0.04 & 0.01 \\
  0.01 & 0.09
\end{pmatrix}
\]

\begin{enumerate}
\item Проверьте гипотезу о незначимости коэффициента $\hb_2$.
\item Найдите предельный эффект роста $x_i$ на вероятность $\P(y_i=1)$ при $x_i=-0.5$.
\item Найдите максимальный предельный эффект роста $x_i$ на вероятность $\P(y_i=1)$.
\item Постройте точечный прогноз вероятности $\P(y_i=1)$ если $x_i = -0.5$.
\item Найдите стандартную ошибку построенного прогноза.
\item Постройте 95\%-ый доверительный интервал для $\P(y_i=1)$ двумя способами (через преобразование интервала для $\hy_i^*$ и через дельта-метод).
\end{enumerate}
\begin{sol}
$z = \frac{\hb_2}{se(\hb_2)}=\frac{3}{0.3}=10$, $H_0$ отвергается.
Предельный эффект равен $\hb_2 \Lambda'(-0.8)\approx 0.642$.
Для нахождения $se(\hat\P)$ найдём линейную аппроксимацию для $\Lambda(\hb_1 + \hb_2 x)$ в окрестности точки $\hb_1=0.7$, $\hb_2=3$. Получаем
\[
\Lambda(\hb_1 + \hb_2 x) \approx \Lambda(\beta_1 + \beta_2 x) + \Lambda'(\beta_1 + \beta_2 x) (\hb_1 - \beta_1) + \Lambda'(\beta_1 + \beta_2 x)x(\hb_2 - \beta_2).
\]
\end{sol}
\end{problem}


\begin{problem}
Почему в пробит-модели предполагается, что $\e_i \sim \cN(0;1)$, а не $\e_i \sim \cN(0;\sigma^2)$ как в линейной регрессии?
\begin{sol}
Если в пробит-уравнении ненаблюдаемой переменной домножить все коэффициенты и стандартную ошибку на произвольную константу, то в результате получится ровно та же модель. Следовательно, модель с $\e_i \sim \cN(0;\sigma^2)$ не идентифицируема. Поэтому надо взять какое-то нормировочное условие. Можно взять, например, $\beta_2 = 42$, но традиционно берут $\e_i \sim \cN(0;1)$.
\end{sol}
\end{problem}



\begin{problem}
Что произойдёт с оценками логит-модели $\P(y_i=1)=F(\beta_1 + \beta_2 x_i)$, их стандартными ошибками, если у зависимой переменной поменять $0$ и $1$ местами?
\begin{sol}

Если посчитать ожидание матрицы Гессе, то оно не зависит от игреков. А значит информация Фишера не зависит от игреков.
Поэтому для подсчёта изменения стандартных ошибок достаточно подставить новые оценки вместо истинных значений параметра.
Можно также воспользоваться тем, что новые оценки являются линейным преобразованием старых.
\end{sol}
\end{problem}


\begin{problem}
  Исследователь Матвей оценил логит-модель по 10 тысячам наблюдений. $\hat\P(y_i = 1) = F(-0.5 + 1.2 x_i)$. Переменная $x_i$ — бинарная, 4 тысячи единиц и 6 тысяч нулей.

\begin{enumerate}
  \item Сколько наблюдений с $y_i = 1$?
  \item Сколько наблюдений с $y_i = 1$ и $x_i = 0$?
  \item Сколько наблюдений с $y_i = 0$ и $x_i = 1$?
\end{enumerate}
\begin{sol}
\end{sol}
\end{problem}


\begin{problem}
Если выбрать покемона наугад, то с вероятностью $p$ покемон окажется ядовитым, а с вероятностью
$(1-p)$ — не ядовитым. Вес ядовитых покемонов распределен нормально, $\cN(\mu_1, \sigma^2)$,
а вес неядовитых — нормально с другим ожиданием и той же дисперсией, $\cN(\mu_2, \sigma^2)$.

Охотник Джон только что поймал покемона и взвесил его.
\begin{enumerate}
    \item Найдите условную вероятность того, что покемон ядовит, если его вес равен $x$.
    \item Запишите найденную условную вероятность в виде
    \[
         \P(\text{покемон ядовит}|x) = F(\beta_1 + \beta_2 x),
    \]
    где $F$ — некая функция распределения. Выразите $\beta_1$ и $\beta_2$ через исходные параметры.
    Как называется функция $F$?
\end{enumerate}
\begin{sol}
    $F(t) = 1/(1+\exp(-t))$.
\end{sol}
\end{problem}


\begin{problem}
Три богатыря в дальнем походе раздобыли набор данных в котором и зависимая переменная и предикторы — бинарные.
Илья Муромец оценивает логит-модель. Добрыня Никитич использует обычный МНК. 
Алёша Попович строит классификационное дерево максимально возможной длины,
используя энтропию для деления узла на два. Как между собой соотносятся прогнозы вероятностей у трёх богатырей?
\begin{sol}
  Они абсолютно равны. Фактически три богатыря просто разбили исходную выборку на части и в качестве 
  прогноза для каждой части наблюдений берут долю игреков, равных единице на этой части выборки.
\end{sol}
\end{problem}



\section{Эндогенность}


\begin{problem}
Величины $x_i$, $z_i$ и $u_i$ имеют совместное распределение, задаваемое табличкой:

\begin{tabular}{ccccc}
  \toprule
  $x_i$ & 0 & 1 & 0 & 1 \\
  $z_i$ & 0 & 1 & 0 & 0 \\
  $u_i$ & -1 & -1 & 1 & 1 \\
  \midrule
  Вероятность & 0.2 & 0.3 & 0.3 & 0.2 \\
  \bottomrule
\end{tabular}

Рассмотрим модель $y_i = \beta x_i + u_i$.

\begin{enumerate}
  \item Найдите $\plim \hb_{OLS}$;
  \item Найдите $\plim \hb_{IV}$, если в качестве инструмента для $x_i$ используется $z_i$;
\end{enumerate}

\begin{sol}
\end{sol}
\end{problem}

\begin{problem}
Рассмотрим три вектора: $y$, $x$ и $z$.
Проведем гипер-плоскость ортогональную $z$ через конец вектора $y$.
Эта гипер-плоскость пересекает прямую порождаемую вектором $x$ в точке $\hb_{IV}x$.

Исходя из данного геометрического определения $\hb_{IV}$:
\begin{enumerate}
  \item Выведите алгоритм двухшагового МНК;
  \item Выведите явную формулу для $\hb_{IV}$;
  \item Докажите, что оценка $\hb_{IV}$ показывает, насколько в среднем растёт $y$ при таком росте $z$, при котором $x$ в среднем растёт на единицу.
\end{enumerate}

\todo[inline]{сказать смысл IV попроще? на две фразы?}
  \begin{sol}
  \end{sol}

\end{problem}


\begin{problem}
 Рассмотрим модель $y_i = \beta x_i + u_i$.
  Исследовательница Мишель строит оценку $\hb_{IV}$ в регрессии $y$ на $x$ с инструментом $z$.
  Исследовательница Аграфена строит обычную МНК оценку в регрессии $\hat y = \hb_{x} x + \hb_{w} w$.

  \begin{enumerate}
    \item Выразите $w$ через $x$, $z$ и $y$ так, чтобы оценка $\hb_{IV}$ Мишель и оценка $\hb_{x}$ Аграфены совпали.
    \item Сформулируйте ещё одну интерпретацию оценки $\hb_{IV}$;
  \end{enumerate}
  \begin{sol}

  \end{sol}
\end{problem}


\begin{problem}
Величины $x_i$, $z_i$ и $u_i$ имеют совместное распределение, параметры которого известны:

\[
  \Var\left(\begin{pmatrix}
    x_i \\
    z_i \\
    u_i \\
\end{pmatrix}\right)=
\begin{pmatrix}
  5 & 1 & -1 \\
  1 & 9 & 0 \\
  -1 & 0 & 4 \\
\end{pmatrix};
\quad
\E\left(\begin{pmatrix}
    x_i \\
    z_i \\
    u_i \\
\end{pmatrix}\right)=
\begin{pmatrix}
4 \\
2 \\
0 \\
\end{pmatrix}
\]

Наблюдения с разными номерами независимы и одинаково распределены.

Рассмотрим модель $y_i =\beta_1 + \beta_2 x_i + u_i$.

\begin{enumerate}
  \item Найдите $\plim \hb_2^{LS}$ и $\plim \hb_1^{LS}$; Являются ли оценки состоятельными?
  \item Храбрый исследователь Афанасий использует двухшаговый МНК.
    На первом шаге он строит регрессию $x_i$ на константу и $z_i$, $\hat x_i = \hat \gamma_1 + \hat \gamma_2 z_i$.
    А на втором регрессию $\hat y_i = \hb_1^{IV} + \hb_2^{IV} \hat x_i$.

    Найдите $\plim \hb_2^{IV}$ и $\plim \hb_1^{IV}$; Являются ли оценки состоятельными?
  \item Как изменятся $\plim \hb_2^{IV}$ и $\plim \hb_1^{IV}$, если Афанасий забудет включить константу на первом шаге?
\end{enumerate}

\begin{sol}
\end{sol}
\end{problem}

\begin{problem}
Приведите примеры дискретных случайных величин $\e$ и $x$, таких, что
\begin{enumerate}
\item $\E(\e)=0$, $\E(\e\mid x)=0$, но величины зависимы. Чему в этом случае равно $\Cov(\e,x)$?
\item $\E(\e)=0$, $\Cov(\e,x)=0$, но $\E(\e\mid x)\neq 0$. Зависимы ли эти случайные величины?
\end{enumerate}


\begin{sol}
\end{sol}
\end{problem}


\begin{problem}
Эконометресса Агнесса хочет оценить модель $y_i=\beta_1+\beta_2 x_i +\e_i$, но, к сожалению, величина $x_i$ ненаблюдаема. Вместо неё доступна величина  $x_i^*$. Величина $x_i^*$  содержит ошибку измерения, $x_i^*=x_i+a_i$. Известно, что $\Var(x_i)=9$, $\Var(a_i)=4$,   $\Var(\e_i)=1$. Величины $x_i$, $a_i$ и $\e_i$ независимы.

Агнесса оценивает регрессию $\hy_i=\hb_1+\hb_2 x_i^*$ с помощью МНК.
\begin{enumerate}
\item Найдите $\plim \hb_2$.
\item Являются ли оценки, получаемые Агнессой, состоятельными?
\end{enumerate}


\begin{sol}
\end{sol}
\end{problem}


\begin{problem}
Эконометресса Анжелла хочет оценить модель $y_i=\beta_1+\beta_2 x_i +\beta_3 w_i +\e_i$, но, к сожалению, величина $w_i$ ненаблюдаема. Известно, что $\Var(x_i)=9$, $\Var(w_i)=4$,  $\Var(\e_i)=1$ и $\Cov(x_i,w_i)=-2$. Случайная составляющая не коррелированна с регрессорами.

За неимением $w_i$ Анжелла оценивает регрессию $\hy_i=\hb_1+\hb_2 x_i$ с помощью МНК.
\begin{enumerate}
\item Найдите $\plim \hb_2$.
\item Являются ли оценки, получаемые Агнессой, состоятельными?
\end{enumerate}

%Анжелла сумела найти переменную $z_i$, такую что $\Cov(z_i,\e_i)=0$ $\Cov(z_i, x_i)=1$


\begin{sol}
\end{sol}
\end{problem}




\begin{problem}
Наблюдения представляют собой случайную выборку. Зависимые переменные $y_{t1}$ и $y_{t2}$ находятся из системы:

\[
\begin{cases}
y_{t1} = \beta_{11} + \beta_{12} x_t + \e_{t1} \\
y_{t2} = \beta_{21} + \beta_{22} z_t + \beta_{23} y_{t1} + \e_{t2}
\end{cases},
\]
где вектор ошибок $\e_t$ имеет совместное нормальное распределение
\[
\e_t \sim \cN\left(
\begin{pmatrix}
  0 \\
  0
\end{pmatrix};
\begin{pmatrix}
  1 & \rho \\
  \rho & 1
\end{pmatrix}
\right)
\]

Эконометресса Анжела оценивает с помощью МНК первое уравнение, а эконометресса Эвридика — второе.
\begin{enumerate}
\item Найдите пределы по вероятности получаемых ими оценок.
\item Будут ли оценки состоятельными?
\end{enumerate}

\begin{sol}
\end{sol}
\end{problem}


\begin{problem}
  Эконометресса Венера оценивает регрессию
  \[
    y_i = \beta_1 + \beta_2 x_i + u_i,
  \]

  А на самом деле $y_i = \beta_1 + \beta_2 x_i + \beta_3 x_i^3 + w_i$, причём $x_i \sim \cN (0; 1)$ и ошибки $w_i \sim \cN(0; \sigma^2)$. Все остальные предпосылки теоремы Гаусса-Маркова выполнены.

  \begin{enumerate}
     \item Будут ли оценки $\hb_1$ и $\hb_2$, получаемые Венерой, состоятельными?
    \item Будут ли оценки $\hb_1$ и $\hb_2$, получаемые Венерой, несмещёнными?
    \item Будут ли состоятельны оценки МНК, если в качестве инструмента Венера возьмёт $z_i = \sin x_i$?
    А $z_i = \cos x_i$?
    \item Как изменятся ответы на предыдущие пункты, если истина имеет вид  $y_i = \beta_1 + \beta_2 x_i + \beta_3 x_i^2 + w_i$?
  \end{enumerate}

\begin{sol}
  Оценки будут несмещёнными и состоятельными. Вызвано это тем, что $u_i = \beta_3 x_i^2 + w_i$ некоррелировано с $x_i$.
\end{sol}
\end{problem}

\begin{problem}
Вектора $(x_1, u_1)$, $(x_2, u_2)$, \ldots независимы и одинаково распределены. Также известно, что $x_i \sim \cN(10; 9)$ и $\E(u_i|x_i)=0$.

Найдите $\plim \left(\frac{1}{n}x'x\right)^{-1}$, $\plim \frac{1}{n}x'u$ и $\plim (x'x)^{-1}x'u$
\begin{sol}
$\plim \left(\frac{1}{n}x'x\right)^{-1} = 109^{-1}$, $\plim \frac{1}{n}x'u = 0$ и $\plim (x'x)^{-1}x'u = 0$
\end{sol}
\end{problem}


\begin{problem}
  Возможно ли, что $\E(x_i | u_i)=0$ и $\E(u_i|x_i)=0$, но при этом $x_i$ и $u_i$ зависимы?
\begin{sol}
  Да, например, равномерное распределение $(u_i, x_i)$ на круге или на окружности. Или равновероятное на восьми точках, $(\pm 1, \pm 1)$, $(\pm 2, \pm 2)$.
\end{sol}
\end{problem}


\begin{problem}
В некотором интституте на некотором факультете задумали провести эксперимент: раздать студентам учебники разных цветов случайным образом и посмотреть на итоговую успеваемость по эконометрике. Учебники есть двух цветов: зелёненькие и красненькие. Поэтому модель имеет вид:
\[
y_i = \beta_1 + \beta_2 green_i + u_i
\]

Здесь $y_i$ — результат по эконометрике, $green_i$ — дамми-переменная на зелёный учебник и $u_i$ — прочие характеристики студента. Зелёные и красненькие учебники планировалось раздавать равновероятно. Однако библиотекарь всё прошляпил и разрешил студентам самим выбирать учебник, какой понравится. В результате вместо переменной $green_i$ получилась переменная $green_i^*$. Известно, что $\E(green_i^*)=\alpha$ и $\Cov(green_i^*, u_i)=\gamma$.

Де-факто оценивалась модель
\[
\hy_i = \hat \theta_1 + \hat \theta_2 green_i^*
\]

\begin{enumerate}
\item Найдите $\plim \hat \theta_1$, $\plim \hat \theta_2$.
\item Найдите $\E \hat \theta_1$, $\E \hat \theta_2$.
\end{enumerate}

\begin{sol}

$\plim \hat \theta_2 = \frac{\gamma}{\alpha (1-\alpha)}$

Заметим, что $\Cov(green_i, green_i^*) = 0$, так как $green_i$ — это результат подбрасывания монетки, $green_i^*$ определяется характеристиками студента.

Ожидание такое же, хотя считается по-другому, $\E(\hat \theta_2) = \frac{\gamma}{\alpha (1-\alpha)}$.

\end{sol}
\end{problem}

\begin{problem}
  Придумайте такие случайные величины $x_1$, $x_2$, $u_1$, что $x_1$ и $x_2$ независимы и одинаково распределены, причём $\E(u_1|x_1)=0$, а $\E(u_1|x_1, x_2) \neq 0$.
\begin{sol}
  Например, можно взять $u_1=x_2$, и все величины $\cN(0;1)$.
\end{sol}
\end{problem}

\begin{problem}
 Рассмотрим классическую парную регрессию со стохастическим регрессором. Всего три наблюдения:

 \begin{tabular}{rr}
 \toprule
 $y_1$ & $x_1$ \\
 $y_2$ & $x_2$ \\
 $y_3$ & $x_3$ \\
 \bottomrule
 \end{tabular}

\begin{enumerate}
 \item Соедините линиями независимые случайные величины.
 \item Соедините линиями одинаково распределённые случайные величины.
\end{enumerate}

\begin{sol}
  Одинаково распределены: $y_1 \sim y_2 \sim y_3$, $x_1 \sim x_2 \sim x_3$. Независимы переменные с разными номерами.
\end{sol}
\end{problem}

\section{GMM}

Прочесть про обобщённый метод моментов можно, например, \cite{zsohar2010short}.

\begin{problem}
Исследователь Максимилиан оценивает параметр $\theta$ с помощью двух моментных условий, $\E(y_i)=2\theta$ и $\E(1/y_i)=\theta$.
С трудом Максимилиан нашёл 200 наблюдений и оказалось, что $\sum 1/y_i = 1.5$.
Сначала Максимилиан оценил $\theta$ с помощью простого метода моментов и первого моментного условия, получил $\hat\theta = 1$.

Затем Максимилиан решил применить обобщённый метод моментов, чтобы учесть оба момента.
В процессе получения GMM оценки Максимилиан обнаружил, что $\Cov(y_i, 1/y_i)=-\theta^2$, $\Var(1/y_i)=9\theta^2$, $\E(y_i^2)=20\theta^2$.

\begin{enumerate}
	\item Найдите ковариационную матрицу моментных условий, $\Var(g)$;
	\item Найдите оптимальную теоретическую матрицу весов $W$;
	\item Оцените матрицу весов $\hat W$;
	\item Найдите оценку GMM с использованием найденной оценки матрицы весов;
\end{enumerate}


\begin{sol}
\[
\Var(g)=\Var\left(
\begin{pmatrix}
y_i - 2\theta \\
1/y_i - \theta\\
\end{pmatrix}
\right)
\]
\end{sol}
\end{problem}




\begin{problem}
Величины $X_i$ равномерны на отрезке $[-a; 3a]$ и независимы. Есть несколько наблюдений, $X_1=0.5$, $X_2=0.7$, $X_3=-0.1$.

\begin{enumerate}
\item Найдите $\E(X_i)$ и $\E(|X_i|)$.
\item Постройте оценку метода моментов, используя $\E(X_i)$.
\item Постройте оценку метода моментов, используя $\E(|X_i|)$.
\item Постройте оценку обобщёного метода моментов используя моменты $\E(X_i)$, $\E(|X_i|)$ и взвешивающую матрицу.
\[
W=\begin{pmatrix}
2 & 0 \\
0 & 1 \\
\end{pmatrix}
\]
\item Найдите оптимальную теоретическую взвешивающую матрицу для обобщённого метода моментов
\item Постройте двухшаговую оценку обобщённого метода моментов, начав со взвешивающей матрицы $W$
\item С помощью полученных оценок постройте 95\%-ый доверительный интервал для неизвестного параметра $a$

\end{enumerate}

\begin{sol}
\end{sol}
\end{problem}




\begin{problem}
Винни-Пух и Пятачок оценивают неизвестный параметр правильности пчёл $\theta$. Когда Винни-Пух проводит очередное измерение параметра правильности, он получает значение $X_i$ нормально распределенное вокруг неизвестного параметра, $X_i \sim \cN(\theta, 1)$. Когда Пятачок проводит измерение параметра правильности, он получает значение $Y_i$, также нормально распределенное вокруг $\theta$, но имеющее большую дисперсию, $Y_i \sim \cN(\theta, 4)$. Различные измерения независимы между собой.
\begin{enumerate}
\item Найдите $\E(X_i)$ и постройте соответствующую оценку метода моментов.
\item Найдите $\E(Y_i)$ и постройте соответствующую оценку метода моментов.
\item Используя два указанных момента найдите обобщённую оценку метода моментов для взвешивающей матрицы
\[
W = \begin{pmatrix}
4 & 0 \\
0 & 9
\end{pmatrix}.
\]
\item Найдите оптимальную взвешивающую матрицу $W$.
\end{enumerate}
\begin{sol}
\end{sol}
\end{problem}


\begin{problem}
Начинающий футболист делает независимые удары по воротам. С вероятностью $\theta$ он попадает левее ворот, с вероятностью $2\theta$ — правее ворот и попадает с вероятностью $1-3\theta$. Из $n$ ударов он попал $N_L$ раз левее ворот и $N_R$ раз — правее.

\begin{enumerate}
\item Найдите $\E(N_L)$ и постройте соответствующую оценку $\theta$ методом моментов.
\item Найдите $\E(N_R)$ и постройте соответствующую оценку $\theta$ методом моментов.
\item Используя два указанных момента постройте оценку обобщённого метода моментов со  взвешивающей матрицей
\[
W = \begin{pmatrix}
4 & 0 \\
0 & 9
\end{pmatrix}.
\]
\item Найдите оптимальную теоретическую взвешивающую матрицу.
\item Для каждой из найденных оценок постройте 95\%-ый доверительный интервал, если $N_L=10$, $N_R=30$, $n=200$.
\end{enumerate}
\begin{sol}
\end{sol}
\end{problem}



\begin{problem}
Можно ли получить МНК-оценки в классической задаче регрессии как оценки обобщённого метода моментов? Можно ли получить оценки метода максимального правдоподобия как оценки обобщённого метода моментов?
\begin{sol}
да, да
\end{sol}
\end{problem}


\begin{problem}
Равшан и Джамшут измеряют длину $\theta$ оставшегося куска рулона обоев много раз.
Измерения Равшана, $X_i$, распределены нормально, $\cN(2\theta, \theta^2 + 100)$.
Измерения Джамшута, $Y_i$, также распределены нормально $\cN(\theta, \theta^2 + 10)$.
Поскольку Равшан и Джамшут спорят друг с другом, их измерения зависимы, $\Cov(X_i, Y_i)=-1$.

	Оказалось, что по 200 (?проверить?) измерениям $\sum X_i=300$, $\sum Y_i = 100$.

Насяльника хочет измерить параметр $\theta$.

\begin{enumerate}
\item Запишите два моментных условия на $\E(X_i)$ и $\E(Y_i)$ в виде
	\[
		\begin{cases}
			\E(g_1(X_i, \theta))=0 \\
			\E(g_2(Y_i, \theta))=0 \\
		\end{cases}
	\]
\item Найдите ковариационную матрицу $\Var(g)$ и теоретическую оптимальную матрицу весов $W$ для обобщённого метода моментов;
\item Найдите оценку параметра $\theta$ методом моментов с единичной матрицей весов;
\item Найдите оценку параметра $\theta$ методом моментов, предварительно оценив оптимальную матрицу с помощью $\hat \theta$ из предыдущего пункта;
\end{enumerate}

\begin{sol}
Оценка при единичной весовой матрице равна $\hat\theta_{W=I}=1.5$.
С точностью до деления на определитель ковариационной матрицы оценка матрицы весов имеет вид:
\[
\hat W =
	\begin{pmatrix}
		12 & 1 \\
		1 & 102 \\
	\end{pmatrix}
\]

\end{sol}
\end{problem}


\section{Панельки}

Хорошо изложена теория у Kurt Schmidtheiny, \url{https://www.schmidheiny.name/teaching/shortguides.htm}.

\begin{problem}
Рассмотрим FE-модель $y_{it} = \alpha + \beta x_{it} + \gamma z_i + c_i + u_{it}$.

Есть всего два момента времени, $t=1$ и $t=2$. Визина использует within-оценку $\hat \beta_{W}$,
а Федя — оценку в первых разностях, $\hat \beta_{FD}$.

\begin{enumerate}
\item Как связаны между собой оценки $\hat \beta_W$ и $\hat \beta_{FD}$?
\item Как связаны между собой $RSS_W$ и $RSS_{FD}$?
\item Являются ли оценки $\hat\beta_W$ и $\hat\beta_{FD}$ состоятельными?
\item Как построить 95\%-ый доверительный интервал с помощью $\hat\beta_{FD}$ и $\hat\beta_W$ для $\beta$?
\end{enumerate}


\begin{sol}
$\hat\beta_W = \hat\beta_{FD}$, $RSS_W = 2RSS_{FD}$, обе оценки состоятельны.
\end{sol}
\end{problem}

\begin{problem}
У храброй исследовательницы Аграфены были панельные данные за три периода, $t=1$, $t=2$, $t=3$.
По старинке для хранения данных Аграфена распечатала все данные на листочках.
Резвый кот Борис безнадёжно испортил листок, относившийся к $t=2$. И тогда Аграфена
решила восстановить данные потерянного периода $t=2$ с помощью линейной аппроксимации, $x_{i2} = (x_{i1} + x_{i3})/2$,
$y_{i2} = (y_{i1} + y_{i3})/2$.

Успешно восстановив испорченные игривым котом Борисом данные, Аграфена приступила к анализу FE-модели
\[
y_{it} = \alpha + \beta x_{it} + \gamma z_i + c_i + u_{it}.
\]

Аграфена оценила коэффициент $\beta$ с помощью within-оценки $\hat\beta_{W}$ и
оценки в первых разностях, $\hat\beta_{FD}$.

\begin{enumerate}
\item Как связаны между собой оценки $\hat \beta_W$ и $\hat \beta_{FD}$?
\item Как связаны между собой $RSS_W$ и $RSS_{FD}$?
\end{enumerate}

\begin{sol}
    $\hat\beta_W = \hat\beta_{FD}$
\end{sol}
\end{problem}



\begin{problem}
Рассмотрим модель $y_{it} = \alpha_i + \beta x_{it} + c_i + u_{it}$, оцениваемую по нескольким точкам:



    \begin{enumerate}
\item Нарисуйте линию сквозной регрессию (pooled ols).
\item Нарисуйте данные в осях первых разностей и нарисуйте линию регрессии в первых разностях.
\item Нарисуйте данные в осях отклонений от среднего и нарисуйте линию within-регрессии.
    \end{enumerate}
\begin{sol}
\end{sol}
\end{problem}

\section{Качественные вопросы}

Не знаю, куда отнести эту тему, но неплохо бы её развить :)

\begin{problem}
Начинающий социолог Аполлон опросил 40 человек. 
В результате у него получилась одна количественная зависимая переменная и 15 регрессоров.
Аполлон хочет построить модель. Что можно посоветовать Аполлону?
\begin{sol}
На 40 наблюдений 15 переменных — это явная переподгонка. Асимптотических свойств МНК 
мы явно не можем использовать. Априори разумно предполагать гетероскедастичность, 
с которой на 40 наблюдениях никак не поборешь. Для использования робастных стандартных ошибок
слишком мало наблюдений. Цель Аполлона слишком размыта. Правильнее было уточнить:
построить модель, чтобы прогнозировать, или построить модель, чтобы проверить,
влияет ли регрессор $z$ на зависимую переменную. 

Вероятно, самое разумное, это применить LASSO выбрав параметр регуляризации так, 
чтобы осталось буквально два-три регрессора. 
\end{sol}
\end{problem}



\section{Большая сила о-малых}

Тут бы хорошо про $o_P$.

\Closesolutionfile{solution_file}


% для гиперссылок на условия
% http://tex.stackexchange.com/questions/45415
\renewenvironment{solution}[1]{%
         % add some glue
         \vskip .5cm plus 2cm minus 0.1cm%
         {\bfseries \hyperlink{problem:#1}{#1.}}%
}%
{%
}%

\section{Решения}
\input{all_solutions}

\section{Источники мудрости}
\printbibliography[heading=none]


\end{document}
